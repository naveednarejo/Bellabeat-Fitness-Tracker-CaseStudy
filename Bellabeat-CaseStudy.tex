% Options for packages loaded elsewhere
\PassOptionsToPackage{unicode}{hyperref}
\PassOptionsToPackage{hyphens}{url}
%
\documentclass[
]{article}
\usepackage{amsmath,amssymb}
\usepackage{lmodern}
\usepackage{iftex}
\ifPDFTeX
  \usepackage[T1]{fontenc}
  \usepackage[utf8]{inputenc}
  \usepackage{textcomp} % provide euro and other symbols
\else % if luatex or xetex
  \usepackage{unicode-math}
  \defaultfontfeatures{Scale=MatchLowercase}
  \defaultfontfeatures[\rmfamily]{Ligatures=TeX,Scale=1}
\fi
% Use upquote if available, for straight quotes in verbatim environments
\IfFileExists{upquote.sty}{\usepackage{upquote}}{}
\IfFileExists{microtype.sty}{% use microtype if available
  \usepackage[]{microtype}
  \UseMicrotypeSet[protrusion]{basicmath} % disable protrusion for tt fonts
}{}
\makeatletter
\@ifundefined{KOMAClassName}{% if non-KOMA class
  \IfFileExists{parskip.sty}{%
    \usepackage{parskip}
  }{% else
    \setlength{\parindent}{0pt}
    \setlength{\parskip}{6pt plus 2pt minus 1pt}}
}{% if KOMA class
  \KOMAoptions{parskip=half}}
\makeatother
\usepackage{xcolor}
\usepackage[margin=1in]{geometry}
\usepackage{color}
\usepackage{fancyvrb}
\newcommand{\VerbBar}{|}
\newcommand{\VERB}{\Verb[commandchars=\\\{\}]}
\DefineVerbatimEnvironment{Highlighting}{Verbatim}{commandchars=\\\{\}}
% Add ',fontsize=\small' for more characters per line
\usepackage{framed}
\definecolor{shadecolor}{RGB}{248,248,248}
\newenvironment{Shaded}{\begin{snugshade}}{\end{snugshade}}
\newcommand{\AlertTok}[1]{\textcolor[rgb]{0.94,0.16,0.16}{#1}}
\newcommand{\AnnotationTok}[1]{\textcolor[rgb]{0.56,0.35,0.01}{\textbf{\textit{#1}}}}
\newcommand{\AttributeTok}[1]{\textcolor[rgb]{0.77,0.63,0.00}{#1}}
\newcommand{\BaseNTok}[1]{\textcolor[rgb]{0.00,0.00,0.81}{#1}}
\newcommand{\BuiltInTok}[1]{#1}
\newcommand{\CharTok}[1]{\textcolor[rgb]{0.31,0.60,0.02}{#1}}
\newcommand{\CommentTok}[1]{\textcolor[rgb]{0.56,0.35,0.01}{\textit{#1}}}
\newcommand{\CommentVarTok}[1]{\textcolor[rgb]{0.56,0.35,0.01}{\textbf{\textit{#1}}}}
\newcommand{\ConstantTok}[1]{\textcolor[rgb]{0.00,0.00,0.00}{#1}}
\newcommand{\ControlFlowTok}[1]{\textcolor[rgb]{0.13,0.29,0.53}{\textbf{#1}}}
\newcommand{\DataTypeTok}[1]{\textcolor[rgb]{0.13,0.29,0.53}{#1}}
\newcommand{\DecValTok}[1]{\textcolor[rgb]{0.00,0.00,0.81}{#1}}
\newcommand{\DocumentationTok}[1]{\textcolor[rgb]{0.56,0.35,0.01}{\textbf{\textit{#1}}}}
\newcommand{\ErrorTok}[1]{\textcolor[rgb]{0.64,0.00,0.00}{\textbf{#1}}}
\newcommand{\ExtensionTok}[1]{#1}
\newcommand{\FloatTok}[1]{\textcolor[rgb]{0.00,0.00,0.81}{#1}}
\newcommand{\FunctionTok}[1]{\textcolor[rgb]{0.00,0.00,0.00}{#1}}
\newcommand{\ImportTok}[1]{#1}
\newcommand{\InformationTok}[1]{\textcolor[rgb]{0.56,0.35,0.01}{\textbf{\textit{#1}}}}
\newcommand{\KeywordTok}[1]{\textcolor[rgb]{0.13,0.29,0.53}{\textbf{#1}}}
\newcommand{\NormalTok}[1]{#1}
\newcommand{\OperatorTok}[1]{\textcolor[rgb]{0.81,0.36,0.00}{\textbf{#1}}}
\newcommand{\OtherTok}[1]{\textcolor[rgb]{0.56,0.35,0.01}{#1}}
\newcommand{\PreprocessorTok}[1]{\textcolor[rgb]{0.56,0.35,0.01}{\textit{#1}}}
\newcommand{\RegionMarkerTok}[1]{#1}
\newcommand{\SpecialCharTok}[1]{\textcolor[rgb]{0.00,0.00,0.00}{#1}}
\newcommand{\SpecialStringTok}[1]{\textcolor[rgb]{0.31,0.60,0.02}{#1}}
\newcommand{\StringTok}[1]{\textcolor[rgb]{0.31,0.60,0.02}{#1}}
\newcommand{\VariableTok}[1]{\textcolor[rgb]{0.00,0.00,0.00}{#1}}
\newcommand{\VerbatimStringTok}[1]{\textcolor[rgb]{0.31,0.60,0.02}{#1}}
\newcommand{\WarningTok}[1]{\textcolor[rgb]{0.56,0.35,0.01}{\textbf{\textit{#1}}}}
\usepackage{graphicx}
\makeatletter
\def\maxwidth{\ifdim\Gin@nat@width>\linewidth\linewidth\else\Gin@nat@width\fi}
\def\maxheight{\ifdim\Gin@nat@height>\textheight\textheight\else\Gin@nat@height\fi}
\makeatother
% Scale images if necessary, so that they will not overflow the page
% margins by default, and it is still possible to overwrite the defaults
% using explicit options in \includegraphics[width, height, ...]{}
\setkeys{Gin}{width=\maxwidth,height=\maxheight,keepaspectratio}
% Set default figure placement to htbp
\makeatletter
\def\fps@figure{htbp}
\makeatother
\setlength{\emergencystretch}{3em} % prevent overfull lines
\providecommand{\tightlist}{%
  \setlength{\itemsep}{0pt}\setlength{\parskip}{0pt}}
\setcounter{secnumdepth}{-\maxdimen} % remove section numbering
\ifLuaTeX
  \usepackage{selnolig}  % disable illegal ligatures
\fi
\IfFileExists{bookmark.sty}{\usepackage{bookmark}}{\usepackage{hyperref}}
\IfFileExists{xurl.sty}{\usepackage{xurl}}{} % add URL line breaks if available
\urlstyle{same} % disable monospaced font for URLs
\hypersetup{
  pdftitle={Bellabeat CaseStudy},
  pdfauthor={Naveed Narejo},
  hidelinks,
  pdfcreator={LaTeX via pandoc}}

\title{Bellabeat CaseStudy}
\author{Naveed Narejo}
\date{2023-02-15}

\begin{document}
\maketitle

\hypertarget{google-data-analytics-capstone-bellabeat-case-study}{%
\section{Google Data Analytics Capstone: Bellabeat Case
Study:}\label{google-data-analytics-capstone-bellabeat-case-study}}

The Bellabeat Marketing Analysis is the second case study of Google's
Data Analytics Professional Certificate program. The case study requires
the analyst to follow to the steps of the data analysis process (Ask,
Prepare, Process, Analyze, Share, and Act) to analyze FitBit Fitness
Tracker dataset.

\hypertarget{step-1-ask}{%
\subsection{Step 1: Ask}\label{step-1-ask}}

\hypertarget{about-the-company}{%
\subsubsection{About the Company :}\label{about-the-company}}

Bellabeat is a high-tech company that manufactures health-focused smart
products. Sršen used her background as an artist to develop beautifully
designed technology that informs and inspires women around the world.
Collecting data on activity, sleep, stress, and reproductive health has
allowed Bellabeat to empower women with knowledge about their own health
and habits.

\hypertarget{key-stakeholders}{%
\subparagraph{Key Stakeholders:}\label{key-stakeholders}}

\begin{itemize}
\tightlist
\item
  Urška Sršen: Bellabeat's co founder and Chief Creative Officer
\item
  Sando Mur: Mathematician and Bellabeat's co-founder
\item
  Bellabeat marketing analytics team: A team of data analysts
  responsible for collecting, analyzing, and reporting data that helps
  guide Bellabeat's marketing strategy
\end{itemize}

\hypertarget{business-task}{%
\subsubsection{Business Task:}\label{business-task}}

Unlock new growth opportunities by analyzing smart device fitness data
and gaining insight into how consumers use their smart devices, which
will help guide he marketing strategy for the company.

\hypertarget{data-analysis-goal}{%
\subparagraph{Data analysis goal:}\label{data-analysis-goal}}

\begin{enumerate}
\def\labelenumi{\arabic{enumi}.}
\tightlist
\item
  What are some trends in smart device usage?
\item
  How could these trends apply to Bellabeat customers?
\item
  How could these trends help influence Bellabeat marketing strategy?
\end{enumerate}

\hypertarget{deliverables}{%
\subsubsection{Deliverables:}\label{deliverables}}

\begin{itemize}
\tightlist
\item
  A clear summary of the business task
\item
  A description of all data sources used
\item
  Documentation of any cleaning or manipulation of data
\item
  A summary of your analysis
\item
  Supporting visualizations and key findings
\item
  Your top high-level content recommendations based on your analysis
\end{itemize}

\hypertarget{step-2-prepare}{%
\subsection{Step 2: Prepare}\label{step-2-prepare}}

Data Source: FitBit Fitness Tracker Data is used for the analysis. It is
an open source data and is available on
Kaggle.\href{https://www.kaggle.com/datasets/arashnic/fitbit}{Click here
for Data.}

The data set contains 18 CSV files organized in long format. Below is a
breakdown of the data using the ROCCC approach: - - Reliability - LOW:
The data comes from 32 users only - Original - LOW: The data is
collected by third party. - Comprehensive - MED: The dataset contains
multiple fields on daily activity intensity, calories used, daily steps
taken, daily sleep time and weight record. - Current - LOW: This data is
from 2016. The data is not current. - Cited - LOW: Data was collected
from a third party, therefore unknown.

\hypertarget{installing-r-package}{%
\subsubsection{Installing R-Package}\label{installing-r-package}}

Installing the R-packages needed for the required analysis.

\begin{Shaded}
\begin{Highlighting}[]
  \CommentTok{\# install.packages("tidyverse")   \#   Functions: dplyr(), tidyr() , stringr(), tibble(), readr(), purrr(), ggplot2() )}
  \CommentTok{\# install.packages("janitor")     \#   Functions: clean\_names(), remove\_empty(), get\_dupes() )}
  \CommentTok{\# install.packages("lubridate")   \#   Functions: datetime fuctions }
  \CommentTok{\# install.packages("plotly")      \#   Functions: plot\_ly()}
  \CommentTok{\# install.packages("dplyr")       \#   Functions: pipes()}
  \CommentTok{\# install.packages("ggalluvial")  \#   Functions: ggplot2 extension }
  \CommentTok{\# install.packages("devtools")    \#   Functions: load\_all(), document(), test(), install\_github()}
  \CommentTok{\# devtools::install\_github("kassambara/ggpubr") \#   Functions: to break in quartiles in graphs}
  \CommentTok{\# install.packages("gginnards")   \#   Functions: ggplot2 extension for static and geometric analysis}
  \CommentTok{\# install.packages("scales")      \#   Functions: labeling as percents, dollars or scientific notation}
\end{Highlighting}
\end{Shaded}

\hypertarget{loading-r-environment}{%
\subsection{Loading R Environment}\label{loading-r-environment}}

Loading the packages needed for the data processing and required
analysis.

\begin{Shaded}
\begin{Highlighting}[]
  \FunctionTok{library}\NormalTok{(tidyverse)}
\end{Highlighting}
\end{Shaded}

\begin{verbatim}
## Warning: package 'tidyverse' was built under R version 4.2.2
\end{verbatim}

\begin{verbatim}
## -- Attaching packages --------------------------------------- tidyverse 1.3.2 --
## v ggplot2 3.3.6      v purrr   0.3.5 
## v tibble  3.1.8      v dplyr   1.0.10
## v tidyr   1.2.1      v stringr 1.4.1 
## v readr   2.1.3      v forcats 0.5.2
\end{verbatim}

\begin{verbatim}
## Warning: package 'dplyr' was built under R version 4.2.2
\end{verbatim}

\begin{verbatim}
## -- Conflicts ------------------------------------------ tidyverse_conflicts() --
## x dplyr::filter() masks stats::filter()
## x dplyr::lag()    masks stats::lag()
\end{verbatim}

\begin{Shaded}
\begin{Highlighting}[]
  \FunctionTok{library}\NormalTok{(janitor)}
\end{Highlighting}
\end{Shaded}

\begin{verbatim}
## Warning: package 'janitor' was built under R version 4.2.2
\end{verbatim}

\begin{verbatim}
## 
## Attaching package: 'janitor'
## 
## The following objects are masked from 'package:stats':
## 
##     chisq.test, fisher.test
\end{verbatim}

\begin{Shaded}
\begin{Highlighting}[]
  \FunctionTok{library}\NormalTok{(lubridate)}
\end{Highlighting}
\end{Shaded}

\begin{verbatim}
## Warning: package 'lubridate' was built under R version 4.2.2
\end{verbatim}

\begin{verbatim}
## Loading required package: timechange
\end{verbatim}

\begin{verbatim}
## Warning: package 'timechange' was built under R version 4.2.2
\end{verbatim}

\begin{verbatim}
## 
## Attaching package: 'lubridate'
## 
## The following objects are masked from 'package:base':
## 
##     date, intersect, setdiff, union
\end{verbatim}

\begin{Shaded}
\begin{Highlighting}[]
  \FunctionTok{library}\NormalTok{(plotly)}
\end{Highlighting}
\end{Shaded}

\begin{verbatim}
## Warning: package 'plotly' was built under R version 4.2.2
\end{verbatim}

\begin{verbatim}
## 
## Attaching package: 'plotly'
## 
## The following object is masked from 'package:ggplot2':
## 
##     last_plot
## 
## The following object is masked from 'package:stats':
## 
##     filter
## 
## The following object is masked from 'package:graphics':
## 
##     layout
\end{verbatim}

\begin{Shaded}
\begin{Highlighting}[]
  \FunctionTok{library}\NormalTok{(dplyr)}
  \FunctionTok{library}\NormalTok{(ggalluvial)}
\end{Highlighting}
\end{Shaded}

\begin{verbatim}
## Warning: package 'ggalluvial' was built under R version 4.2.2
\end{verbatim}

\begin{Shaded}
\begin{Highlighting}[]
  \FunctionTok{library}\NormalTok{(ggpubr)}
  \FunctionTok{library}\NormalTok{(gghighlight)}
\end{Highlighting}
\end{Shaded}

\begin{verbatim}
## Warning: package 'gghighlight' was built under R version 4.2.2
\end{verbatim}

\begin{Shaded}
\begin{Highlighting}[]
  \FunctionTok{library}\NormalTok{(gginnards)}
\end{Highlighting}
\end{Shaded}

\begin{verbatim}
## Warning: package 'gginnards' was built under R version 4.2.2
\end{verbatim}

\begin{Shaded}
\begin{Highlighting}[]
  \FunctionTok{library}\NormalTok{(scales)}
\end{Highlighting}
\end{Shaded}

\begin{verbatim}
## Warning: package 'scales' was built under R version 4.2.2
\end{verbatim}

\begin{verbatim}
## 
## Attaching package: 'scales'
## 
## The following object is masked from 'package:purrr':
## 
##     discard
## 
## The following object is masked from 'package:readr':
## 
##     col_factor
\end{verbatim}

\hypertarget{step-3-process}{%
\subsection{Step 3: Process}\label{step-3-process}}

\hypertarget{importing-files}{%
\subsubsection{Importing files}\label{importing-files}}

For this analysis we will be using these four datasets:

\begin{itemize}
\tightlist
\item
  dailyActivity\_merged.csv
\item
  sleepDay\_merged.csv
\item
  weightLogInfo\_merged.csv
\item
  hourlysteps\_merged.csv
\end{itemize}

\begin{Shaded}
\begin{Highlighting}[]
\NormalTok{ daily\_activity }\OtherTok{\textless{}{-}} \FunctionTok{read\_csv}\NormalTok{(}\AttributeTok{file =} \StringTok{"D:}\SpecialCharTok{\textbackslash{}\textbackslash{}}\StringTok{Fitness Tracker CaseStudy}\SpecialCharTok{\textbackslash{}\textbackslash{}}\StringTok{Fitabase Data 4.12.16{-}5.12.16}\SpecialCharTok{\textbackslash{}\textbackslash{}}\StringTok{dailyActivity\_merged.csv"}\NormalTok{)}\SpecialCharTok{\%\textgreater{}\%}
 \FunctionTok{remove\_empty}\NormalTok{()}
\end{Highlighting}
\end{Shaded}

\begin{verbatim}
## value for "which" not specified, defaulting to c("rows", "cols")
\end{verbatim}

\begin{verbatim}
## Rows: 940 Columns: 16
## -- Column specification --------------------------------------------------------
## Delimiter: ","
## chr  (2): ActivityDate, Day
## dbl (14): Id, TotalSteps, TotalDistance, TrackerDistance, LoggedActivitiesDi...
## 
## i Use `spec()` to retrieve the full column specification for this data.
## i Specify the column types or set `show_col_types = FALSE` to quiet this message.
\end{verbatim}

\begin{Shaded}
\begin{Highlighting}[]
\NormalTok{ steps\_perhour }\OtherTok{\textless{}{-}} \FunctionTok{read\_csv}\NormalTok{(}\AttributeTok{file =} \StringTok{"D:}\SpecialCharTok{\textbackslash{}\textbackslash{}}\StringTok{Fitness Tracker CaseStudy}\SpecialCharTok{\textbackslash{}\textbackslash{}}\StringTok{Fitabase Data 4.12.16{-}5.12.16}\SpecialCharTok{\textbackslash{}\textbackslash{}}\StringTok{hourlySteps\_merged.csv"}\NormalTok{)}\SpecialCharTok{\%\textgreater{}\%}
 \FunctionTok{remove\_empty}\NormalTok{()}
\end{Highlighting}
\end{Shaded}

\begin{verbatim}
## value for "which" not specified, defaulting to c("rows", "cols")
## Rows: 22099 Columns: 4-- Column specification --------------------------------------------------------
## Delimiter: ","
## chr  (1): ActivityHour
## dbl  (2): Id, StepTotal
## time (1): Hours
## i Use `spec()` to retrieve the full column specification for this data.
## i Specify the column types or set `show_col_types = FALSE` to quiet this message.
\end{verbatim}

\begin{Shaded}
\begin{Highlighting}[]
\NormalTok{ sleep\_data }\OtherTok{\textless{}{-}} \FunctionTok{read\_csv}\NormalTok{(}\AttributeTok{file =} \StringTok{"D:}\SpecialCharTok{\textbackslash{}\textbackslash{}}\StringTok{Fitness Tracker CaseStudy}\SpecialCharTok{\textbackslash{}\textbackslash{}}\StringTok{Fitabase Data 4.12.16{-}5.12.16}\SpecialCharTok{\textbackslash{}\textbackslash{}}\StringTok{sleepDay\_merged.csv"}\NormalTok{)}\SpecialCharTok{\%\textgreater{}\%}
 \FunctionTok{remove\_empty}\NormalTok{()}
\end{Highlighting}
\end{Shaded}

\begin{verbatim}
## value for "which" not specified, defaulting to c("rows", "cols")
## Rows: 413 Columns: 5-- Column specification --------------------------------------------------------
## Delimiter: ","
## chr (1): SleepDay
## dbl (4): Id, TotalSleepRecords, TotalMinutesAsleep, TotalTimeInBed
## i Use `spec()` to retrieve the full column specification for this data.
## i Specify the column types or set `show_col_types = FALSE` to quiet this message.
\end{verbatim}

\begin{Shaded}
\begin{Highlighting}[]
\NormalTok{  weight\_data }\OtherTok{\textless{}{-}} \FunctionTok{read\_csv}\NormalTok{(}\AttributeTok{file =} \StringTok{"D:}\SpecialCharTok{\textbackslash{}\textbackslash{}}\StringTok{Fitness Tracker CaseStudy}\SpecialCharTok{\textbackslash{}\textbackslash{}}\StringTok{Fitabase Data 4.12.16{-}5.12.16}\SpecialCharTok{\textbackslash{}\textbackslash{}}\StringTok{weightLogInfo\_merged.csv"}\NormalTok{)}\SpecialCharTok{\%\textgreater{}\%}
 \FunctionTok{remove\_empty}\NormalTok{()}
\end{Highlighting}
\end{Shaded}

\begin{verbatim}
## value for "which" not specified, defaulting to c("rows", "cols")
## Rows: 67 Columns: 8-- Column specification --------------------------------------------------------
## Delimiter: ","
## chr (1): Date
## dbl (6): Id, WeightKg, WeightPounds, Fat, BMI, LogId
## lgl (1): IsManualReport
## i Use `spec()` to retrieve the full column specification for this data.
## i Specify the column types or set `show_col_types = FALSE` to quiet this message.
\end{verbatim}

\begin{Shaded}
\begin{Highlighting}[]
     \FunctionTok{head}\NormalTok{(daily\_activity) }\CommentTok{\# 1st glimpse of the data}
\end{Highlighting}
\end{Shaded}

\begin{verbatim}
## # A tibble: 6 x 16
##         Id Activ~1 Day   Total~2 Total~3 Track~4 Logge~5 VeryA~6 Moder~7 Light~8
##      <dbl> <chr>   <chr>   <dbl>   <dbl>   <dbl>   <dbl>   <dbl>   <dbl>   <dbl>
## 1   1.50e9 4/12/2~ Tues~   13162    8.5     8.5        0    1.88   0.550    6.06
## 2   1.50e9 4/13/2~ Wedn~   10735    6.97    6.97       0    1.57   0.690    4.71
## 3   1.50e9 4/14/2~ Thur~   10460    6.74    6.74       0    2.44   0.400    3.91
## 4   1.50e9 4/15/2~ Frid~    9762    6.28    6.28       0    2.14   1.26     2.83
## 5   1.50e9 4/16/2~ Satu~   12669    8.16    8.16       0    2.71   0.410    5.04
## 6   1.50e9 4/17/2~ Sund~    9705    6.48    6.48       0    3.19   0.780    2.51
## # ... with 6 more variables: SedentaryActiveDistance <dbl>,
## #   VeryActiveMinutes <dbl>, FairlyActiveMinutes <dbl>,
## #   LightlyActiveMinutes <dbl>, SedentaryMinutes <dbl>, Calories <dbl>, and
## #   abbreviated variable names 1: ActivityDate, 2: TotalSteps,
## #   3: TotalDistance, 4: TrackerDistance, 5: LoggedActivitiesDistance,
## #   6: VeryActiveDistance, 7: ModeratelyActiveDistance, 8: LightActiveDistance
\end{verbatim}

\begin{Shaded}
\begin{Highlighting}[]
     \FunctionTok{head}\NormalTok{(weight\_data) }\CommentTok{\# 1st glimpse of the data}
\end{Highlighting}
\end{Shaded}

\begin{verbatim}
## # A tibble: 6 x 8
##           Id Date                  WeightKg Weight~1   Fat   BMI IsMan~2   LogId
##        <dbl> <chr>                    <dbl>    <dbl> <dbl> <dbl> <lgl>     <dbl>
## 1 1503960366 5/2/2016 11:59:59 PM      52.6     116.    22  22.6 TRUE    1.46e12
## 2 1503960366 5/3/2016 11:59:59 PM      52.6     116.    NA  22.6 TRUE    1.46e12
## 3 1927972279 4/13/2016 1:08:52 AM     134.      294.    NA  47.5 FALSE   1.46e12
## 4 2873212765 4/21/2016 11:59:59 PM     56.7     125.    NA  21.5 TRUE    1.46e12
## 5 2873212765 5/12/2016 11:59:59 PM     57.3     126.    NA  21.7 TRUE    1.46e12
## 6 4319703577 4/17/2016 11:59:59 PM     72.4     160.    25  27.5 TRUE    1.46e12
## # ... with abbreviated variable names 1: WeightPounds, 2: IsManualReport
\end{verbatim}

\begin{Shaded}
\begin{Highlighting}[]
     \FunctionTok{head}\NormalTok{(sleep\_data) }\CommentTok{\# 1st glimpse of the data}
\end{Highlighting}
\end{Shaded}

\begin{verbatim}
## # A tibble: 6 x 5
##           Id SleepDay              TotalSleepRecords TotalMinutesAsleep TotalT~1
##        <dbl> <chr>                             <dbl>              <dbl>    <dbl>
## 1 1503960366 4/12/2016 12:00:00 AM                 1                327      346
## 2 1503960366 4/13/2016 12:00:00 AM                 2                384      407
## 3 1503960366 4/15/2016 12:00:00 AM                 1                412      442
## 4 1503960366 4/16/2016 12:00:00 AM                 2                340      367
## 5 1503960366 4/17/2016 12:00:00 AM                 1                700      712
## 6 1503960366 4/19/2016 12:00:00 AM                 1                304      320
## # ... with abbreviated variable name 1: TotalTimeInBed
\end{verbatim}

\begin{Shaded}
\begin{Highlighting}[]
     \FunctionTok{head}\NormalTok{(weight\_data) }\CommentTok{\# 1st glimpse of the data}
\end{Highlighting}
\end{Shaded}

\begin{verbatim}
## # A tibble: 6 x 8
##           Id Date                  WeightKg Weight~1   Fat   BMI IsMan~2   LogId
##        <dbl> <chr>                    <dbl>    <dbl> <dbl> <dbl> <lgl>     <dbl>
## 1 1503960366 5/2/2016 11:59:59 PM      52.6     116.    22  22.6 TRUE    1.46e12
## 2 1503960366 5/3/2016 11:59:59 PM      52.6     116.    NA  22.6 TRUE    1.46e12
## 3 1927972279 4/13/2016 1:08:52 AM     134.      294.    NA  47.5 FALSE   1.46e12
## 4 2873212765 4/21/2016 11:59:59 PM     56.7     125.    NA  21.5 TRUE    1.46e12
## 5 2873212765 5/12/2016 11:59:59 PM     57.3     126.    NA  21.7 TRUE    1.46e12
## 6 4319703577 4/17/2016 11:59:59 PM     72.4     160.    25  27.5 TRUE    1.46e12
## # ... with abbreviated variable names 1: WeightPounds, 2: IsManualReport
\end{verbatim}

To make sure that everything required is imported correctly, we need to
verify by using View() and head() functions. Now that I've imported all
of the data frames that I'll be using, I can start cleaning up the data.
I will look at each data frame to familiarize myself with the data and
check for errors.I'll be analyzing each file separately.

\hypertarget{cleaning-formatting-organizing-daily-activity-dataframe}{%
\subparagraph{\texorpdfstring{\textbf{Cleaning, Formatting \&
Organizing: Daily Activity
dataframe}}{Cleaning, Formatting \& Organizing: Daily Activity dataframe}}\label{cleaning-formatting-organizing-daily-activity-dataframe}}

\begin{Shaded}
\begin{Highlighting}[]
 \FunctionTok{View}\NormalTok{(daily\_activity) }\CommentTok{\#viewing the complete dataset}
 \FunctionTok{str}\NormalTok{(daily\_activity) }\CommentTok{\#inspecting structure of data i.e data types}
\end{Highlighting}
\end{Shaded}

\begin{verbatim}
## tibble [940 x 16] (S3: tbl_df/tbl/data.frame)
##  $ Id                      : num [1:940] 1.5e+09 1.5e+09 1.5e+09 1.5e+09 1.5e+09 ...
##  $ ActivityDate            : chr [1:940] "4/12/2016" "4/13/2016" "4/14/2016" "4/15/2016" ...
##  $ Day                     : chr [1:940] "Tuesday" "Wednesday" "Thursday" "Friday" ...
##  $ TotalSteps              : num [1:940] 13162 10735 10460 9762 12669 ...
##  $ TotalDistance           : num [1:940] 8.5 6.97 6.74 6.28 8.16 ...
##  $ TrackerDistance         : num [1:940] 8.5 6.97 6.74 6.28 8.16 ...
##  $ LoggedActivitiesDistance: num [1:940] 0 0 0 0 0 0 0 0 0 0 ...
##  $ VeryActiveDistance      : num [1:940] 1.88 1.57 2.44 2.14 2.71 ...
##  $ ModeratelyActiveDistance: num [1:940] 0.55 0.69 0.4 1.26 0.41 ...
##  $ LightActiveDistance     : num [1:940] 6.06 4.71 3.91 2.83 5.04 ...
##  $ SedentaryActiveDistance : num [1:940] 0 0 0 0 0 0 0 0 0 0 ...
##  $ VeryActiveMinutes       : num [1:940] 25 21 30 29 36 38 42 50 28 19 ...
##  $ FairlyActiveMinutes     : num [1:940] 13 19 11 34 10 20 16 31 12 8 ...
##  $ LightlyActiveMinutes    : num [1:940] 328 217 181 209 221 164 233 264 205 211 ...
##  $ SedentaryMinutes        : num [1:940] 728 776 1218 726 773 ...
##  $ Calories                : num [1:940] 1985 1797 1776 1745 1863 ...
\end{verbatim}

\begin{Shaded}
\begin{Highlighting}[]
 \FunctionTok{colnames}\NormalTok{(daily\_activity)  }\CommentTok{\#checking column names}
\end{Highlighting}
\end{Shaded}

\begin{verbatim}
##  [1] "Id"                       "ActivityDate"            
##  [3] "Day"                      "TotalSteps"              
##  [5] "TotalDistance"            "TrackerDistance"         
##  [7] "LoggedActivitiesDistance" "VeryActiveDistance"      
##  [9] "ModeratelyActiveDistance" "LightActiveDistance"     
## [11] "SedentaryActiveDistance"  "VeryActiveMinutes"       
## [13] "FairlyActiveMinutes"      "LightlyActiveMinutes"    
## [15] "SedentaryMinutes"         "Calories"
\end{verbatim}

\begin{Shaded}
\begin{Highlighting}[]
\NormalTok{ daily\_activity }\OtherTok{\textless{}{-}} \FunctionTok{clean\_names}\NormalTok{(daily\_activity) }\CommentTok{\# cleaning column names to snake format}
 \FunctionTok{get\_dupes}\NormalTok{(daily\_activity) }\CommentTok{\#no duplicate found}
\end{Highlighting}
\end{Shaded}

\begin{verbatim}
## No variable names specified - using all columns.
\end{verbatim}

\begin{verbatim}
## No duplicate combinations found of: id, activity_date, day, total_steps, total_distance, tracker_distance, logged_activities_distance, very_active_distance, moderately_active_distance, ... and 7 other variables
\end{verbatim}

\begin{verbatim}
## # A tibble: 0 x 17
## # ... with 17 variables: id <dbl>, activity_date <chr>, day <chr>,
## #   total_steps <dbl>, total_distance <dbl>, tracker_distance <dbl>,
## #   logged_activities_distance <dbl>, very_active_distance <dbl>,
## #   moderately_active_distance <dbl>, light_active_distance <dbl>,
## #   sedentary_active_distance <dbl>, very_active_minutes <dbl>,
## #   fairly_active_minutes <dbl>, lightly_active_minutes <dbl>,
## #   sedentary_minutes <dbl>, calories <dbl>, dupe_count <int>
\end{verbatim}

\begin{Shaded}
\begin{Highlighting}[]
\NormalTok{ daily\_activity}\SpecialCharTok{$}\NormalTok{activity\_date }\OtherTok{\textless{}{-}} \FunctionTok{as.Date}\NormalTok{(}\FunctionTok{strptime}\NormalTok{(daily\_activity}\SpecialCharTok{$}\NormalTok{activity\_date, }\StringTok{"\%m/\%d/\%Y"}\NormalTok{)) }\CommentTok{\# Changing dates to date format. Note: Date here is character, however it should be date}
 \FunctionTok{str}\NormalTok{(daily\_activity) }\CommentTok{\# rechecking data for changes verification}
\end{Highlighting}
\end{Shaded}

\begin{verbatim}
## tibble [940 x 16] (S3: tbl_df/tbl/data.frame)
##  $ id                        : num [1:940] 1.5e+09 1.5e+09 1.5e+09 1.5e+09 1.5e+09 ...
##  $ activity_date             : Date[1:940], format: "2016-04-12" "2016-04-13" ...
##  $ day                       : chr [1:940] "Tuesday" "Wednesday" "Thursday" "Friday" ...
##  $ total_steps               : num [1:940] 13162 10735 10460 9762 12669 ...
##  $ total_distance            : num [1:940] 8.5 6.97 6.74 6.28 8.16 ...
##  $ tracker_distance          : num [1:940] 8.5 6.97 6.74 6.28 8.16 ...
##  $ logged_activities_distance: num [1:940] 0 0 0 0 0 0 0 0 0 0 ...
##  $ very_active_distance      : num [1:940] 1.88 1.57 2.44 2.14 2.71 ...
##  $ moderately_active_distance: num [1:940] 0.55 0.69 0.4 1.26 0.41 ...
##  $ light_active_distance     : num [1:940] 6.06 4.71 3.91 2.83 5.04 ...
##  $ sedentary_active_distance : num [1:940] 0 0 0 0 0 0 0 0 0 0 ...
##  $ very_active_minutes       : num [1:940] 25 21 30 29 36 38 42 50 28 19 ...
##  $ fairly_active_minutes     : num [1:940] 13 19 11 34 10 20 16 31 12 8 ...
##  $ lightly_active_minutes    : num [1:940] 328 217 181 209 221 164 233 264 205 211 ...
##  $ sedentary_minutes         : num [1:940] 728 776 1218 726 773 ...
##  $ calories                  : num [1:940] 1985 1797 1776 1745 1863 ...
\end{verbatim}

\begin{Shaded}
\begin{Highlighting}[]
\NormalTok{daily\_activity }\SpecialCharTok{\%\textgreater{}\%}
\FunctionTok{summarise}\NormalTok{(}\AttributeTok{total\_users =} \FunctionTok{n\_distinct}\NormalTok{(daily\_activity}\SpecialCharTok{$}\NormalTok{id)) }\CommentTok{\#total number of unique users}
\end{Highlighting}
\end{Shaded}

\begin{verbatim}
## # A tibble: 1 x 1
##   total_users
##         <int>
## 1          33
\end{verbatim}

\begin{Shaded}
\begin{Highlighting}[]
\NormalTok{daily\_activity }\SpecialCharTok{\%\textgreater{}\%}
\FunctionTok{select}\NormalTok{(total\_steps,}
\NormalTok{total\_distance,}
\NormalTok{calories) }\SpecialCharTok{\%\textgreater{}\%}
\FunctionTok{summary}\NormalTok{()}
\end{Highlighting}
\end{Shaded}

\begin{verbatim}
##   total_steps    total_distance      calories   
##  Min.   :    0   Min.   : 0.000   Min.   :   0  
##  1st Qu.: 3790   1st Qu.: 2.620   1st Qu.:1828  
##  Median : 7406   Median : 5.245   Median :2134  
##  Mean   : 7638   Mean   : 5.490   Mean   :2304  
##  3rd Qu.:10727   3rd Qu.: 7.713   3rd Qu.:2793  
##  Max.   :36019   Max.   :28.030   Max.   :4900
\end{verbatim}

In the above chunks the Daily Activity dataframe is analyzed for blanks
NA, Data type error, Irrelevant/Inconsistent formatting and duplicate.
There were no duplicates found the dataframe, all the column names are
converted in to snake\_format, and data type of DATE column changed from
Character to Date. Here we also familiarize ourselves with the data a
bit more i.e.~we look for Min/Avg/Max distance traveled, steps walked
and calories burned by all the users during 31 days.

\hypertarget{cleaning-formatting-organizing-steps-per-hour-dataframe}{%
\subparagraph{\texorpdfstring{\textbf{Cleaning, Formatting \&
Organizing: Steps Per Hour
dataframe}}{Cleaning, Formatting \& Organizing: Steps Per Hour dataframe}}\label{cleaning-formatting-organizing-steps-per-hour-dataframe}}

\begin{Shaded}
\begin{Highlighting}[]
 \FunctionTok{View}\NormalTok{(steps\_perhour) }\CommentTok{\#viewing the complete dataset}
 \FunctionTok{str}\NormalTok{(steps\_perhour) }\CommentTok{\#inspecting structure of data i.e data types. Note: Date here is character, however it should be date}
\end{Highlighting}
\end{Shaded}

\begin{verbatim}
## tibble [22,099 x 4] (S3: tbl_df/tbl/data.frame)
##  $ Id          : num [1:22099] 1.5e+09 1.5e+09 1.5e+09 1.5e+09 1.5e+09 ...
##  $ ActivityHour: chr [1:22099] "4/12/2016 0:00" "4/12/2016 1:00" "4/12/2016 2:00" "4/12/2016 3:00" ...
##  $ Hours       : 'hms' num [1:22099] 00:00:00 01:00:00 02:00:00 03:00:00 ...
##   ..- attr(*, "units")= chr "secs"
##  $ StepTotal   : num [1:22099] 373 160 151 0 0 ...
\end{verbatim}

\begin{Shaded}
\begin{Highlighting}[]
 \FunctionTok{colnames}\NormalTok{(steps\_perhour)  }\CommentTok{\#checking column names}
\end{Highlighting}
\end{Shaded}

\begin{verbatim}
## [1] "Id"           "ActivityHour" "Hours"        "StepTotal"
\end{verbatim}

\begin{Shaded}
\begin{Highlighting}[]
\NormalTok{ steps\_perhour }\OtherTok{\textless{}{-}} \FunctionTok{clean\_names}\NormalTok{(steps\_perhour) }\CommentTok{\# cleaning column names to snake format}
 \FunctionTok{get\_dupes}\NormalTok{(steps\_perhour) }\CommentTok{\#no duplicate found}
\end{Highlighting}
\end{Shaded}

\begin{verbatim}
## No variable names specified - using all columns.
\end{verbatim}

\begin{verbatim}
## No duplicate combinations found of: id, activity_hour, hours, step_total
\end{verbatim}

\begin{verbatim}
## # A tibble: 0 x 5
## # ... with 5 variables: id <dbl>, activity_hour <chr>, hours <time>,
## #   step_total <dbl>, dupe_count <int>
\end{verbatim}

\begin{Shaded}
\begin{Highlighting}[]
\NormalTok{ steps\_perhour}\SpecialCharTok{$}\NormalTok{activity\_hour }\OtherTok{\textless{}{-}} \FunctionTok{as.Date}\NormalTok{(steps\_perhour}\SpecialCharTok{$}\NormalTok{activity\_hour, }\StringTok{"\%m/\%d/\%Y"}\NormalTok{)  }\CommentTok{\# Changing dates to date format. Note: Date here is character, however it should be date}
 \FunctionTok{str}\NormalTok{(steps\_perhour) }\CommentTok{\# rechecking data for changes verification}
\end{Highlighting}
\end{Shaded}

\begin{verbatim}
## tibble [22,099 x 4] (S3: tbl_df/tbl/data.frame)
##  $ id           : num [1:22099] 1.5e+09 1.5e+09 1.5e+09 1.5e+09 1.5e+09 ...
##  $ activity_hour: Date[1:22099], format: "2016-04-12" "2016-04-12" ...
##  $ hours        : 'hms' num [1:22099] 00:00:00 01:00:00 02:00:00 03:00:00 ...
##   ..- attr(*, "units")= chr "secs"
##  $ step_total   : num [1:22099] 373 160 151 0 0 ...
\end{verbatim}

\begin{Shaded}
\begin{Highlighting}[]
\NormalTok{steps\_perhour }\SpecialCharTok{\%\textgreater{}\%}
\FunctionTok{summarise}\NormalTok{(}\AttributeTok{total\_users =} \FunctionTok{n\_distinct}\NormalTok{(steps\_perhour}\SpecialCharTok{$}\NormalTok{id)) }\CommentTok{\#total number of unique users}
\end{Highlighting}
\end{Shaded}

\begin{verbatim}
## # A tibble: 1 x 1
##   total_users
##         <int>
## 1          33
\end{verbatim}

\begin{Shaded}
\begin{Highlighting}[]
\NormalTok{steps\_perhour }\SpecialCharTok{\%\textgreater{}\%}
\FunctionTok{select}\NormalTok{(step\_total) }\SpecialCharTok{\%\textgreater{}\%}
\FunctionTok{summary}\NormalTok{()}
\end{Highlighting}
\end{Shaded}

\begin{verbatim}
##    step_total     
##  Min.   :    0.0  
##  1st Qu.:    0.0  
##  Median :   40.0  
##  Mean   :  320.2  
##  3rd Qu.:  357.0  
##  Max.   :10554.0
\end{verbatim}

In the above chunks the steps per hour dataframe is analyzed for blanks
NA, Data type error, Irrelevant/Inconsistent formatting and duplicate.
There were no duplicates found the dataframe, all the column names are
converted in to snake\_format, and data type of DATE column changed from
Character to Date. Here we also familiarize ourselves with the data a
bit more i.e.~we look for Min/Avg/Max steps taken by the users.

\hypertarget{cleaning-formatting-organizing-sleep-dataframe}{%
\subparagraph{\texorpdfstring{\textbf{Cleaning, Formatting \&
Organizing: Sleep
dataframe}}{Cleaning, Formatting \& Organizing: Sleep dataframe}}\label{cleaning-formatting-organizing-sleep-dataframe}}

\begin{Shaded}
\begin{Highlighting}[]
 \FunctionTok{View}\NormalTok{(sleep\_data) }\CommentTok{\#viewing the complete dataset}
 \FunctionTok{str}\NormalTok{(sleep\_data) }\CommentTok{\#inspecting structure of data i.e data types. Note: Date here is character, however it should be date}
\end{Highlighting}
\end{Shaded}

\begin{verbatim}
## tibble [413 x 5] (S3: tbl_df/tbl/data.frame)
##  $ Id                : num [1:413] 1.5e+09 1.5e+09 1.5e+09 1.5e+09 1.5e+09 ...
##  $ SleepDay          : chr [1:413] "4/12/2016 12:00:00 AM" "4/13/2016 12:00:00 AM" "4/15/2016 12:00:00 AM" "4/16/2016 12:00:00 AM" ...
##  $ TotalSleepRecords : num [1:413] 1 2 1 2 1 1 1 1 1 1 ...
##  $ TotalMinutesAsleep: num [1:413] 327 384 412 340 700 304 360 325 361 430 ...
##  $ TotalTimeInBed    : num [1:413] 346 407 442 367 712 320 377 364 384 449 ...
\end{verbatim}

\begin{Shaded}
\begin{Highlighting}[]
 \FunctionTok{colnames}\NormalTok{(sleep\_data)  }\CommentTok{\#checking column names}
\end{Highlighting}
\end{Shaded}

\begin{verbatim}
## [1] "Id"                 "SleepDay"           "TotalSleepRecords" 
## [4] "TotalMinutesAsleep" "TotalTimeInBed"
\end{verbatim}

\begin{Shaded}
\begin{Highlighting}[]
\NormalTok{ sleep\_data }\OtherTok{\textless{}{-}} \FunctionTok{clean\_names}\NormalTok{(sleep\_data) }\CommentTok{\# cleaning column names to snake format}
\NormalTok{ sleep\_data}\SpecialCharTok{$}\NormalTok{sleep\_day }\OtherTok{\textless{}{-}} \FunctionTok{as.Date}\NormalTok{(}\FunctionTok{strptime}\NormalTok{(sleep\_data}\SpecialCharTok{$}\NormalTok{sleep\_day, }\StringTok{"\%m/\%d/\%Y"}\NormalTok{)) }\CommentTok{\# Changing dates to date format. Note: Date here is character, however it should be date}
 \FunctionTok{get\_dupes}\NormalTok{(sleep\_data) }\CommentTok{\# 6 duplicate found }
\end{Highlighting}
\end{Shaded}

\begin{verbatim}
## No variable names specified - using all columns.
\end{verbatim}

\begin{verbatim}
## # A tibble: 6 x 6
##           id sleep_day  total_sleep_records total_minutes_asleep total~1 dupe_~2
##        <dbl> <date>                   <dbl>                <dbl>   <dbl>   <int>
## 1 4388161847 2016-05-05                   1                  471     495       2
## 2 4388161847 2016-05-05                   1                  471     495       2
## 3 4702921684 2016-05-07                   1                  520     543       2
## 4 4702921684 2016-05-07                   1                  520     543       2
## 5 8378563200 2016-04-25                   1                  388     402       2
## 6 8378563200 2016-04-25                   1                  388     402       2
## # ... with abbreviated variable names 1: total_time_in_bed, 2: dupe_count
\end{verbatim}

\begin{Shaded}
\begin{Highlighting}[]
\NormalTok{ sleep\_duplicates }\OtherTok{\textless{}{-}} \FunctionTok{get\_dupes}\NormalTok{(sleep\_data) }\CommentTok{\# duplicates saved to remove later}
\end{Highlighting}
\end{Shaded}

\begin{verbatim}
## No variable names specified - using all columns.
\end{verbatim}

\begin{Shaded}
\begin{Highlighting}[]
 \FunctionTok{str}\NormalTok{(sleep\_data) }\CommentTok{\# rechecking data for changes verification}
\end{Highlighting}
\end{Shaded}

\begin{verbatim}
## tibble [413 x 5] (S3: tbl_df/tbl/data.frame)
##  $ id                  : num [1:413] 1.5e+09 1.5e+09 1.5e+09 1.5e+09 1.5e+09 ...
##  $ sleep_day           : Date[1:413], format: "2016-04-12" "2016-04-13" ...
##  $ total_sleep_records : num [1:413] 1 2 1 2 1 1 1 1 1 1 ...
##  $ total_minutes_asleep: num [1:413] 327 384 412 340 700 304 360 325 361 430 ...
##  $ total_time_in_bed   : num [1:413] 346 407 442 367 712 320 377 364 384 449 ...
\end{verbatim}

\begin{Shaded}
\begin{Highlighting}[]
 \CommentTok{\#removing duplicates using anti{-}join}
\NormalTok{ sleep\_data }\OtherTok{\textless{}{-}} \FunctionTok{anti\_join}\NormalTok{(sleep\_data, sleep\_duplicates)}
\end{Highlighting}
\end{Shaded}

\begin{verbatim}
## Joining, by = c("id", "sleep_day", "total_sleep_records",
## "total_minutes_asleep", "total_time_in_bed")
\end{verbatim}

\begin{Shaded}
\begin{Highlighting}[]
 \FunctionTok{rm}\NormalTok{(sleep\_duplicates)}
\end{Highlighting}
\end{Shaded}

\begin{Shaded}
\begin{Highlighting}[]
\NormalTok{sleep\_data }\SpecialCharTok{\%\textgreater{}\%}
\FunctionTok{summarise}\NormalTok{(}\AttributeTok{total\_users =} \FunctionTok{n\_distinct}\NormalTok{(sleep\_data}\SpecialCharTok{$}\NormalTok{id)) }\CommentTok{\#total number of unique users}
\end{Highlighting}
\end{Shaded}

\begin{verbatim}
## # A tibble: 1 x 1
##   total_users
##         <int>
## 1          24
\end{verbatim}

\begin{Shaded}
\begin{Highlighting}[]
\NormalTok{sleep\_data }\SpecialCharTok{\%\textgreater{}\%}
\FunctionTok{select}\NormalTok{(}
\NormalTok{total\_minutes\_asleep,}
\NormalTok{total\_time\_in\_bed) }\SpecialCharTok{\%\textgreater{}\%}
\FunctionTok{summary}\NormalTok{()}
\end{Highlighting}
\end{Shaded}

\begin{verbatim}
##  total_minutes_asleep total_time_in_bed
##  Min.   : 58.0        Min.   : 61.0    
##  1st Qu.:361.0        1st Qu.:404.5    
##  Median :432.0        Median :463.0    
##  Mean   :418.9        Mean   :458.3    
##  3rd Qu.:490.0        3rd Qu.:526.0    
##  Max.   :796.0        Max.   :961.0
\end{verbatim}

In the above chunks the sleep dataframe is analyzed for its blanks, NA,
Data type error, Irrelevant/Inconsistent formatting and duplicate. We
looked for duplicates and a total of 6 duplicates found, I saved it as a
separate data frame and removed the duplicate by using Anti\_Join. Here
all the column names are converted in to snake\_format, and data type of
DATE column changed from Character to Date. Here we also familiarize
ourselves with the data a bit more i.e.~We look for Min/Avg/Max
minutes\_asleep and time\_in\_bed.

\hypertarget{cleaning-formatting-organizing-weight-dataframe}{%
\subparagraph{\texorpdfstring{\textbf{Cleaning, Formatting \&
Organizing: Weight
dataframe}}{Cleaning, Formatting \& Organizing: Weight dataframe}}\label{cleaning-formatting-organizing-weight-dataframe}}

\begin{Shaded}
\begin{Highlighting}[]
 \FunctionTok{View}\NormalTok{(weight\_data) }\CommentTok{\#viewing the complete dataset}
 \FunctionTok{str}\NormalTok{(weight\_data) }\CommentTok{\#inspecting structure of data i.e data types. Note: Date here is character, however it should be date}
\end{Highlighting}
\end{Shaded}

\begin{verbatim}
## tibble [67 x 8] (S3: tbl_df/tbl/data.frame)
##  $ Id            : num [1:67] 1.50e+09 1.50e+09 1.93e+09 2.87e+09 2.87e+09 ...
##  $ Date          : chr [1:67] "5/2/2016 11:59:59 PM" "5/3/2016 11:59:59 PM" "4/13/2016 1:08:52 AM" "4/21/2016 11:59:59 PM" ...
##  $ WeightKg      : num [1:67] 52.6 52.6 133.5 56.7 57.3 ...
##  $ WeightPounds  : num [1:67] 116 116 294 125 126 ...
##  $ Fat           : num [1:67] 22 NA NA NA NA 25 NA NA NA NA ...
##  $ BMI           : num [1:67] 22.6 22.6 47.5 21.5 21.7 ...
##  $ IsManualReport: logi [1:67] TRUE TRUE FALSE TRUE TRUE TRUE ...
##  $ LogId         : num [1:67] 1.46e+12 1.46e+12 1.46e+12 1.46e+12 1.46e+12 ...
\end{verbatim}

\begin{Shaded}
\begin{Highlighting}[]
 \FunctionTok{colnames}\NormalTok{(weight\_data)  }\CommentTok{\#checking column names}
\end{Highlighting}
\end{Shaded}

\begin{verbatim}
## [1] "Id"             "Date"           "WeightKg"       "WeightPounds"  
## [5] "Fat"            "BMI"            "IsManualReport" "LogId"
\end{verbatim}

\begin{Shaded}
\begin{Highlighting}[]
\NormalTok{ weight\_data }\OtherTok{\textless{}{-}} \FunctionTok{clean\_names}\NormalTok{(weight\_data) }\CommentTok{\# cleaning column names to snake format}
\NormalTok{ weight\_data}\SpecialCharTok{$}\NormalTok{date }\OtherTok{\textless{}{-}} \FunctionTok{as.Date}\NormalTok{(}\FunctionTok{strptime}\NormalTok{(weight\_data}\SpecialCharTok{$}\NormalTok{date, }\StringTok{"\%m/\%d/\%Y"}\NormalTok{))  }\CommentTok{\# Changing dates to date format. Note: Date here is character, however it should be date}
 \FunctionTok{get\_dupes}\NormalTok{(weight\_data) }\CommentTok{\# no duplicate found }
\end{Highlighting}
\end{Shaded}

\begin{verbatim}
## No variable names specified - using all columns.
\end{verbatim}

\begin{verbatim}
## No duplicate combinations found of: id, date, weight_kg, weight_pounds, fat, bmi, is_manual_report, log_id
\end{verbatim}

\begin{verbatim}
## # A tibble: 0 x 9
## # ... with 9 variables: id <dbl>, date <date>, weight_kg <dbl>,
## #   weight_pounds <dbl>, fat <dbl>, bmi <dbl>, is_manual_report <lgl>,
## #   log_id <dbl>, dupe_count <int>
\end{verbatim}

\begin{Shaded}
\begin{Highlighting}[]
 \FunctionTok{str}\NormalTok{(weight\_data) }\CommentTok{\# rechecking data for changes verification}
\end{Highlighting}
\end{Shaded}

\begin{verbatim}
## tibble [67 x 8] (S3: tbl_df/tbl/data.frame)
##  $ id              : num [1:67] 1.50e+09 1.50e+09 1.93e+09 2.87e+09 2.87e+09 ...
##  $ date            : Date[1:67], format: "2016-05-02" "2016-05-03" ...
##  $ weight_kg       : num [1:67] 52.6 52.6 133.5 56.7 57.3 ...
##  $ weight_pounds   : num [1:67] 116 116 294 125 126 ...
##  $ fat             : num [1:67] 22 NA NA NA NA 25 NA NA NA NA ...
##  $ bmi             : num [1:67] 22.6 22.6 47.5 21.5 21.7 ...
##  $ is_manual_report: logi [1:67] TRUE TRUE FALSE TRUE TRUE TRUE ...
##  $ log_id          : num [1:67] 1.46e+12 1.46e+12 1.46e+12 1.46e+12 1.46e+12 ...
\end{verbatim}

\begin{Shaded}
\begin{Highlighting}[]
\NormalTok{weight\_data }\SpecialCharTok{\%\textgreater{}\%}
\FunctionTok{summarise}\NormalTok{(}\AttributeTok{total\_users =} \FunctionTok{n\_distinct}\NormalTok{(weight\_data}\SpecialCharTok{$}\NormalTok{id)) }\CommentTok{\#total number of unique users}
\end{Highlighting}
\end{Shaded}

\begin{verbatim}
## # A tibble: 1 x 1
##   total_users
##         <int>
## 1           8
\end{verbatim}

\begin{Shaded}
\begin{Highlighting}[]
\NormalTok{weight\_data }\SpecialCharTok{\%\textgreater{}\%}
\FunctionTok{select}\NormalTok{(}
\NormalTok{weight\_kg,}
\NormalTok{weight\_pounds,}
\NormalTok{fat) }\SpecialCharTok{\%\textgreater{}\%}
\FunctionTok{summary}\NormalTok{()}
\end{Highlighting}
\end{Shaded}

\begin{verbatim}
##    weight_kg      weight_pounds        fat       
##  Min.   : 52.60   Min.   :116.0   Min.   :22.00  
##  1st Qu.: 61.40   1st Qu.:135.4   1st Qu.:22.75  
##  Median : 62.50   Median :137.8   Median :23.50  
##  Mean   : 72.04   Mean   :158.8   Mean   :23.50  
##  3rd Qu.: 85.05   3rd Qu.:187.5   3rd Qu.:24.25  
##  Max.   :133.50   Max.   :294.3   Max.   :25.00  
##                                   NA's   :65
\end{verbatim}

In the above chunks the weight dataframe is analyzed for blanks, NA,
Data type error, Irrelevant/Inconsistent formatting and duplicate.There
were no duplicates found the dataframe, all the column names are
converted in to snake\_format, and data type of DATE column changed from
Character to Date. Here we also familiarize ourselves with the data a
bit more i.e.~we look for Min/Avg/Max weight in KGs and pounds.

\hypertarget{step-4-5-analyze-share}{%
\subsection{Step 4-5: Analyze \& Share}\label{step-4-5-analyze-share}}

\hypertarget{analyzing-steps-data---creating-total-steps-histograms}{%
\subparagraph{\texorpdfstring{\textbf{Analyzing: Steps Data - Creating
Total Steps
Histograms}}{Analyzing: Steps Data - Creating Total Steps Histograms}}\label{analyzing-steps-data---creating-total-steps-histograms}}

Creating a histogram to display the distribution of steps taken each day

\begin{Shaded}
\begin{Highlighting}[]
\NormalTok{steps\_graph\_with\_outlier }\OtherTok{\textless{}{-}} \FunctionTok{ggplot}\NormalTok{(daily\_activity, }\FunctionTok{aes}\NormalTok{(}\AttributeTok{x=}\NormalTok{total\_steps)) }\SpecialCharTok{+} 
      \FunctionTok{geom\_histogram}\NormalTok{(}\FunctionTok{aes}\NormalTok{(}\AttributeTok{y =}\NormalTok{ ..density..), }\AttributeTok{colour =} \StringTok{"\#1F3552"}\NormalTok{, }\AttributeTok{fill =} \StringTok{"\#D8E2EB"}\NormalTok{) }\SpecialCharTok{+} 
      \FunctionTok{geom\_density}\NormalTok{() }\SpecialCharTok{+}
      \FunctionTok{theme\_cleveland}\NormalTok{ () }\SpecialCharTok{+}
      \FunctionTok{scale\_y\_continuous}\NormalTok{(}\AttributeTok{labels =}\NormalTok{ scientific)}

\NormalTok{steps\_graph\_with\_outlier}
\end{Highlighting}
\end{Shaded}

\begin{verbatim}
## `stat_bin()` using `bins = 30`. Pick better value with `binwidth`.
\end{verbatim}

\includegraphics{Bellabeat-CaseStudy_files/figure-latex/steps_graph: Steps Analysis after Outlier removal-1.pdf}

\begin{Shaded}
\begin{Highlighting}[]
\CommentTok{\#cleaning the environment, dropping the obsolete dataframe}
\FunctionTok{rm}\NormalTok{(steps\_graph\_with\_outlier)}
\end{Highlighting}
\end{Shaded}

This is a normal looking curve with a positive skew and extreme outliers
after 25,000 steps Now, I Filtered out days where no activity is
recorded and outliers and rerun save the results, I will remove the
outliers and create a new graph.

\begin{Shaded}
\begin{Highlighting}[]
\CommentTok{\#Removing outliers}

\NormalTok{daily\_activity }\OtherTok{\textless{}{-}}\NormalTok{ daily\_activity[(daily\_activity}\SpecialCharTok{$}\NormalTok{total\_steps}\SpecialCharTok{\textgreater{}}\DecValTok{0}\NormalTok{) }\SpecialCharTok{\&}\NormalTok{ (daily\_activity}\SpecialCharTok{$}\NormalTok{total\_steps }\SpecialCharTok{\textless{}} \DecValTok{25000}\NormalTok{) ,]}

\CommentTok{\#steps\_graph: Steps Analysis  after removing outliers}
\CommentTok{\# Produce new histogram}

\NormalTok{steps\_graph\_without\_outliers }\OtherTok{\textless{}{-}} 
  \FunctionTok{ggplot}\NormalTok{(daily\_activity, }\FunctionTok{aes}\NormalTok{(total\_steps,)) }\SpecialCharTok{+} 
  \FunctionTok{geom\_histogram}\NormalTok{(}\FunctionTok{aes}\NormalTok{(}\AttributeTok{y =}\NormalTok{ ..density..), }\AttributeTok{colour =} \StringTok{"\#1F3552"}\NormalTok{, }\AttributeTok{fill =} \StringTok{"\#c8c7d7"}\NormalTok{) }\SpecialCharTok{+} 
  \FunctionTok{geom\_density}\NormalTok{(}\AttributeTok{alpha =} \FloatTok{0.3}\NormalTok{) }\SpecialCharTok{+}
  \FunctionTok{ggtitle}\NormalTok{(}\StringTok{"Historgram of Total Steps"}\NormalTok{) }\SpecialCharTok{+} 
  \FunctionTok{theme\_cleveland}\NormalTok{ () }\SpecialCharTok{+}
  \FunctionTok{labs}\NormalTok{(}\AttributeTok{title =} \StringTok{"Total Steps Historgram"}\NormalTok{, }
       \AttributeTok{x =} \StringTok{"Total Steps"}\NormalTok{, }\AttributeTok{y =} \StringTok{"Density"}\NormalTok{)}

\NormalTok{steps\_graph\_without\_outliers}
\end{Highlighting}
\end{Shaded}

\begin{verbatim}
## `stat_bin()` using `bins = 30`. Pick better value with `binwidth`.
\end{verbatim}

\includegraphics{Bellabeat-CaseStudy_files/figure-latex/steps_graph: Steps Analysis-1.pdf}

\begin{Shaded}
\begin{Highlighting}[]
\CommentTok{\#cleaning the environment, dropping the obsolete dataframe}
\FunctionTok{rm}\NormalTok{(steps\_graph\_without\_outliers)}
\end{Highlighting}
\end{Shaded}

There is still a skew, but we have minimized extreme bias.

\hypertarget{running-descriptive-analysis}{%
\subparagraph{\texorpdfstring{\textbf{Running Descriptive
Analysis}}{Running Descriptive Analysis}}\label{running-descriptive-analysis}}

Here, I am running a descriptive statistics to familiarize myself with
the data.

\begin{Shaded}
\begin{Highlighting}[]
\NormalTok{monthly\_user\_activity }\OtherTok{\textless{}{-}}
\NormalTok{      daily\_activity  }\SpecialCharTok{\%\textgreater{}\%} 
      \FunctionTok{group\_by}\NormalTok{(id) }\SpecialCharTok{\%\textgreater{}\%} 
      \FunctionTok{summarise}\NormalTok{(}
        \AttributeTok{record\_count =} \FunctionTok{n}\NormalTok{(),}
        \AttributeTok{sum\_steps\_monthly =} \FunctionTok{sum}\NormalTok{(total\_steps),}
        \AttributeTok{min\_steps\_monthly =} \FunctionTok{min}\NormalTok{(total\_steps),}
        \AttributeTok{max\_steps\_monthly =} \FunctionTok{max}\NormalTok{(total\_steps),}
        \AttributeTok{median\_steps\_monthly =} \FunctionTok{median}\NormalTok{(total\_steps),}
        \AttributeTok{median\_steps\_daily =} \FunctionTok{median}\NormalTok{(total\_steps)}\SpecialCharTok{/}\FunctionTok{n}\NormalTok{(),}
        \AttributeTok{avg\_steps\_monthly =} \FunctionTok{mean}\NormalTok{(total\_steps),}
        \AttributeTok{std\_steps\_monthly =} \FunctionTok{sd}\NormalTok{(total\_steps)}
\NormalTok{      )}

\NormalTok{monthly\_user\_activity}
\end{Highlighting}
\end{Shaded}

\begin{verbatim}
## # A tibble: 33 x 9
##            id record_c~1 sum_s~2 min_s~3 max_s~4 media~5 media~6 avg_s~7 std_s~8
##         <dbl>      <int>   <dbl>   <dbl>   <dbl>   <dbl>   <dbl>   <dbl>   <dbl>
##  1 1503960366         30  375619    9705   18134  12438    415.   12521.   2099.
##  2 1624580081         30  142042    1510   10536   3726.   124.    4735.   2610.
##  3 1644430081         30  218489    1223   18213   6684.   223.    7283.   4325.
##  4 1844505072         21   79982       4    8054   4014    191.    3809.   2475.
##  5 1927972279         17   28400     149    3790   1675     98.5   1671.   1177.
##  6 2022484408         31  352490    3292   18387  11548    373.   11371.   2807.
##  7 2026352035         31  172573     254   12357   5528    178.    5567.   2978.
##  8 2320127002         31  146223     772   10725   5057    163.    4717.   2255.
##  9 2347167796         18  171354      42   22244   9781    543.    9520.   4682.
## 10 2873212765         31  234229    2524    9685   7762    250.    7556.   1514.
## # ... with 23 more rows, and abbreviated variable names 1: record_count,
## #   2: sum_steps_monthly, 3: min_steps_monthly, 4: max_steps_monthly,
## #   5: median_steps_monthly, 6: median_steps_daily, 7: avg_steps_monthly,
## #   8: std_steps_monthly
\end{verbatim}

Some individuals have extremely low engagement, which may skew analyses.
I'll group people according to their levels of engagement in order to
analyze it further.

Now for detailed analysis, I'm distributing the participants into four
groups (Never, Low, Moderate, and High) on how often they track their
health. It will help us to analyze the Weight data,Steps count and Sleep
activities for each user. The users with more then 20 days records are
categorized as \textbf{High}, the ones with activity records between 10
and 20 days are assigned the \textbf{Moderate} category, users with
records less then 10 days are categorized as \textbf{Low} and for those
who have NO records for any activity are categorized as \textbf{Never}.

\begin{Shaded}
\begin{Highlighting}[]
\NormalTok{daily\_steps\_records }\OtherTok{\textless{}{-}} 
\NormalTok{        daily\_activity }\SpecialCharTok{\%\textgreater{}\%}
        \FunctionTok{count}\NormalTok{(id) }\SpecialCharTok{\%\textgreater{}\%}
        \FunctionTok{mutate}\NormalTok{(}\AttributeTok{steps\_groups =} \FunctionTok{case\_when}\NormalTok{(}
\NormalTok{          n }\SpecialCharTok{\textless{}=} \DecValTok{10} \SpecialCharTok{\textasciitilde{}} \StringTok{"Low"}\NormalTok{,}
\NormalTok{          n }\SpecialCharTok{\textless{}=} \DecValTok{20} \SpecialCharTok{\textasciitilde{}} \StringTok{"Moderate"}\NormalTok{,}
\NormalTok{          n }\SpecialCharTok{\textless{}=} \DecValTok{31} \SpecialCharTok{\textasciitilde{}} \StringTok{"High"}
\NormalTok{        )}
\NormalTok{           )}

       
      \CommentTok{\#counting how many nights each participant recorded their sleep and also grouping them}
       \CommentTok{\#based on frequency of recording}
\NormalTok{      sleep\_records }\OtherTok{\textless{}{-}}
\NormalTok{        sleep\_data }\SpecialCharTok{\%\textgreater{}\%}
        \FunctionTok{count}\NormalTok{(id) }\SpecialCharTok{\%\textgreater{}\%}
        \FunctionTok{mutate}\NormalTok{(}\AttributeTok{sleep\_groups =} \FunctionTok{case\_when}\NormalTok{(}
\NormalTok{          n }\SpecialCharTok{\textless{}=} \DecValTok{10} \SpecialCharTok{\textasciitilde{}} \StringTok{"Low"}\NormalTok{,}
\NormalTok{          n }\SpecialCharTok{\textless{}=} \DecValTok{20} \SpecialCharTok{\textasciitilde{}} \StringTok{"Moderate"}\NormalTok{,}
\NormalTok{          n }\SpecialCharTok{\textless{}=} \DecValTok{31} \SpecialCharTok{\textasciitilde{}} \StringTok{"High"}
\NormalTok{        )}
\NormalTok{        )}

      
       \CommentTok{\#doing the same for weight}
\NormalTok{      weight\_records }\OtherTok{\textless{}{-}}
\NormalTok{        weight\_data }\SpecialCharTok{\%\textgreater{}\%}
        \FunctionTok{count}\NormalTok{(id) }\SpecialCharTok{\%\textgreater{}\%}
        \FunctionTok{mutate}\NormalTok{(}\AttributeTok{weight\_groups =} \FunctionTok{case\_when}\NormalTok{(}
\NormalTok{          n }\SpecialCharTok{\textless{}=} \DecValTok{10} \SpecialCharTok{\textasciitilde{}} \StringTok{"Low"}\NormalTok{,}
\NormalTok{          n }\SpecialCharTok{\textless{}=} \DecValTok{20} \SpecialCharTok{\textasciitilde{}} \StringTok{"Moderate"}\NormalTok{,}
\NormalTok{          n }\SpecialCharTok{\textless{}=} \DecValTok{31} \SpecialCharTok{\textasciitilde{}} \StringTok{"High"}
\NormalTok{        )}
\NormalTok{        )}
      
    
      
\CommentTok{\#full join with sleep}
\NormalTok{daily\_steps\_records }\OtherTok{\textless{}{-}} 
\NormalTok{  daily\_steps\_records       }\SpecialCharTok{\%\textgreater{}\%} 
  \FunctionTok{full\_join}\NormalTok{(sleep\_records, }\AttributeTok{by=}\StringTok{"id"}\NormalTok{)}
  
\CommentTok{\# full join with weight}
\NormalTok{daily\_steps\_records }\OtherTok{\textless{}{-}} 
\NormalTok{  daily\_steps\_records       }\SpecialCharTok{\%\textgreater{}\%} 
  \FunctionTok{full\_join}\NormalTok{(weight\_records, }\AttributeTok{by=}\StringTok{"id"}\NormalTok{) }

\NormalTok{daily\_steps\_records }\OtherTok{\textless{}{-}} 
\NormalTok{  daily\_steps\_records }\SpecialCharTok{\%\textgreater{}\%}
  \FunctionTok{replace\_na}\NormalTok{(}\FunctionTok{list}\NormalTok{(}\AttributeTok{sleep\_records =} \DecValTok{0}\NormalTok{, }\AttributeTok{weight\_records =} \DecValTok{0}\NormalTok{, }\AttributeTok{sleep\_groups =} \StringTok{"Never"}\NormalTok{, }\AttributeTok{weight\_groups =} \StringTok{"Never"}\NormalTok{))}


\CommentTok{\#renaming the column 2 of the data set}
\FunctionTok{colnames}\NormalTok{(daily\_steps\_records)[}\DecValTok{2}\NormalTok{] }\OtherTok{\textless{}{-}} \StringTok{\textquotesingle{}total\_count\textquotesingle{}}


\CommentTok{\#Dropping extra columns created while merge}
\NormalTok{daily\_steps\_records }\OtherTok{\textless{}{-}}\NormalTok{ daily\_steps\_records[ }\SpecialCharTok{{-}}\FunctionTok{c}\NormalTok{(}\DecValTok{4}\NormalTok{,}\DecValTok{6}\NormalTok{) ]}

      
      
  \CommentTok{\#categorizing overall device usage by the participants    }
\NormalTok{    sleep\_groups\_percentage }\OtherTok{\textless{}{-}}\NormalTok{   daily\_steps\_records }\SpecialCharTok{\%\textgreater{}\%} \FunctionTok{group\_by}\NormalTok{(sleep\_groups) }\SpecialCharTok{\%\textgreater{}\%} \FunctionTok{summarise}\NormalTok{(}\AttributeTok{Percentage=}\FunctionTok{n}\NormalTok{()}\SpecialCharTok{/}\FunctionTok{nrow}\NormalTok{(.)) }\CommentTok{\#calculating the \% of High,Low,Med for users}
      \FunctionTok{colnames}\NormalTok{(sleep\_groups\_percentage)[}\DecValTok{1}\NormalTok{] }\OtherTok{\textless{}{-}} \StringTok{\textquotesingle{}group\textquotesingle{}}
    
\NormalTok{    steps\_groups\_percentage }\OtherTok{\textless{}{-}}\NormalTok{   daily\_steps\_records }\SpecialCharTok{\%\textgreater{}\%} \FunctionTok{group\_by}\NormalTok{(steps\_groups) }\SpecialCharTok{\%\textgreater{}\%} \FunctionTok{summarise}\NormalTok{(}\AttributeTok{Percentage=}\FunctionTok{n}\NormalTok{()}\SpecialCharTok{/}\FunctionTok{nrow}\NormalTok{(.))  }\CommentTok{\#calculating the \% of High,Low,Med for users}
          \FunctionTok{colnames}\NormalTok{(steps\_groups\_percentage)[}\DecValTok{1}\NormalTok{] }\OtherTok{\textless{}{-}} \StringTok{\textquotesingle{}group\textquotesingle{}}

\NormalTok{    weight\_groups\_percentage }\OtherTok{\textless{}{-}}\NormalTok{  daily\_steps\_records }\SpecialCharTok{\%\textgreater{}\%} \FunctionTok{group\_by}\NormalTok{(weight\_groups) }\SpecialCharTok{\%\textgreater{}\%} \FunctionTok{summarise}\NormalTok{(}\AttributeTok{Percentage=}\FunctionTok{n}\NormalTok{()}\SpecialCharTok{/}\FunctionTok{nrow}\NormalTok{(.))  }\CommentTok{\#calculating the \% of High,Low,Med for users}
          \FunctionTok{colnames}\NormalTok{(weight\_groups\_percentage)[}\DecValTok{1}\NormalTok{] }\OtherTok{\textless{}{-}} \StringTok{\textquotesingle{}group\textquotesingle{}}

\CommentTok{\#joining all activities in a single dataframe.}
\NormalTok{device\_user\_percentage }\OtherTok{\textless{}{-}} 
\NormalTok{   steps\_groups\_percentage    }\SpecialCharTok{\%\textgreater{}\%} 
  \FunctionTok{full\_join}\NormalTok{(sleep\_groups\_percentage, }\AttributeTok{by=}\StringTok{"group"}\NormalTok{)}


\NormalTok{device\_user\_percentage }\OtherTok{\textless{}{-}} 
\NormalTok{  device\_user\_percentage       }\SpecialCharTok{\%\textgreater{}\%} 
  \FunctionTok{full\_join}\NormalTok{(weight\_groups\_percentage, }\AttributeTok{by=}\StringTok{"group"}\NormalTok{)}
    
\FunctionTok{colnames}\NormalTok{(device\_user\_percentage) }\OtherTok{\textless{}{-}} \FunctionTok{c}\NormalTok{(}\StringTok{\textquotesingle{}Recording\_Frequency\textquotesingle{}}\NormalTok{,}\StringTok{\textquotesingle{}steps\_Percent\textquotesingle{}}\NormalTok{,}\StringTok{\textquotesingle{}sleep\_Percent\textquotesingle{}}\NormalTok{, }\StringTok{\textquotesingle{}weight\_percent\textquotesingle{}}\NormalTok{) }\CommentTok{\#renaming the dataframe columns}
\NormalTok{device\_user\_percentage[}\FunctionTok{is.na}\NormalTok{(device\_user\_percentage)] }\OtherTok{=} \DecValTok{0} \CommentTok{\#replacing all NAs with 0}
\NormalTok{device\_user\_percentage[}\DecValTok{2}\SpecialCharTok{:}\DecValTok{4}\NormalTok{] }\OtherTok{\textless{}{-}} \FunctionTok{sapply}\NormalTok{(device\_user\_percentage[}\DecValTok{2}\SpecialCharTok{:}\DecValTok{4}\NormalTok{], }\ControlFlowTok{function}\NormalTok{(x) }\FunctionTok{percent}\NormalTok{(x, }\AttributeTok{accuracy=}\DecValTok{2}\NormalTok{)) }\CommentTok{\#converting decimal values to \% form.}


\CommentTok{\#cleaning the environment, dropping all the obsolete dataframes}
\FunctionTok{rm}\NormalTok{(sleep\_records, weight\_records,sleep\_groups\_percentage,steps\_groups\_percentage,weight\_groups\_percentage)     }


\NormalTok{daily\_steps\_records  }
\end{Highlighting}
\end{Shaded}

\begin{verbatim}
## # A tibble: 33 x 5
##            id total_count steps_groups sleep_groups weight_groups
##         <dbl>       <int> <chr>        <chr>        <chr>        
##  1 1503960366          30 High         High         Low          
##  2 1624580081          30 High         Never        Never        
##  3 1644430081          30 High         Low          Never        
##  4 1844505072          21 High         Low          Never        
##  5 1927972279          17 Moderate     Low          Low          
##  6 2022484408          31 High         Never        Never        
##  7 2026352035          31 High         High         Never        
##  8 2320127002          31 High         Low          Never        
##  9 2347167796          18 Moderate     Moderate     Never        
## 10 2873212765          31 High         Never        Low          
## # ... with 23 more rows
\end{verbatim}

\hypertarget{activity-wise-device-engagement---alluvial-plot}{%
\subparagraph{\texorpdfstring{\textbf{Activity wise Device Engagement -
Alluvial
plot}}{Activity wise Device Engagement - Alluvial plot}}\label{activity-wise-device-engagement---alluvial-plot}}

In order to determine which features of the devices are most and least
used, we now look for the device conversion rate or engagement rate.
This will make it easier for us to comprehend how user numbers
increase/decrease as users switch between different activities

\begin{Shaded}
\begin{Highlighting}[]
\CommentTok{\#Transforming the recording frequency data frame into a long format to make the building of the alluvial plot easier.}

\NormalTok{actvities\_graph\_data  }\OtherTok{\textless{}{-}}\NormalTok{ daily\_steps\_records[}\FunctionTok{c}\NormalTok{(}\StringTok{"id"}\NormalTok{, }\StringTok{"steps\_groups"}\NormalTok{, }\StringTok{"sleep\_groups"}\NormalTok{, }\StringTok{"weight\_groups"}\NormalTok{)] }\SpecialCharTok{\%\textgreater{}\%}
  \FunctionTok{pivot\_longer}\NormalTok{(}\SpecialCharTok{!}\NormalTok{id )}

  

\NormalTok{actvties\_analysis\_graph  }\OtherTok{\textless{}{-}} 
    \FunctionTok{ggplot}\NormalTok{(actvities\_graph\_data, }\FunctionTok{aes}\NormalTok{(}\AttributeTok{x =}\NormalTok{ name, }\AttributeTok{stratum =}\NormalTok{ value, }\AttributeTok{alluvium =}\NormalTok{ id, }\AttributeTok{fill =}\NormalTok{ value, }\AttributeTok{label =}\NormalTok{ value)) }\SpecialCharTok{+}
    \FunctionTok{scale\_x\_discrete}\NormalTok{(}\AttributeTok{limits =} \FunctionTok{c}\NormalTok{(}\StringTok{"steps\_groups"}\NormalTok{, }\StringTok{"sleep\_groups"}\NormalTok{, }\StringTok{"weight\_groups"}\NormalTok{), }\AttributeTok{expand =} \FunctionTok{c}\NormalTok{(.}\DecValTok{1}\NormalTok{, .}\DecValTok{1}\NormalTok{))}\SpecialCharTok{+}
    \FunctionTok{geom\_flow}\NormalTok{() }\SpecialCharTok{+}
    \FunctionTok{geom\_stratum}\NormalTok{(}\AttributeTok{alpha =}\NormalTok{ .}\DecValTok{5}\NormalTok{) }\SpecialCharTok{+}
    \FunctionTok{geom\_text}\NormalTok{(}\AttributeTok{stat =} \StringTok{"stratum"}\NormalTok{, }\AttributeTok{size =} \DecValTok{3}\NormalTok{) }\SpecialCharTok{+}
    \FunctionTok{labs}\NormalTok{(}\AttributeTok{title =} \StringTok{"Device Usage for Each actvity"}\NormalTok{, }\AttributeTok{x =} \StringTok{" "}\NormalTok{, }\AttributeTok{y =} \StringTok{"Number of users"}\NormalTok{) }\SpecialCharTok{+}
   \FunctionTok{scale\_y\_reverse}\NormalTok{()}\SpecialCharTok{+}
    \FunctionTok{scale\_fill\_brewer}\NormalTok{(}\AttributeTok{type =} \StringTok{"qual"}\NormalTok{, }\AttributeTok{palette =} \StringTok{"Set2"}\NormalTok{) }\SpecialCharTok{+}
    \FunctionTok{theme}\NormalTok{(}\AttributeTok{legend.position =} \StringTok{"none"}\NormalTok{, }
          \AttributeTok{panel.background =} \FunctionTok{element\_blank}\NormalTok{(),}
          \AttributeTok{axis.text.x =} \FunctionTok{element\_text}\NormalTok{(}\AttributeTok{size =} \DecValTok{14}\NormalTok{),}
          \AttributeTok{axis.text.y =} \FunctionTok{element\_text}\NormalTok{(}\AttributeTok{size =} \DecValTok{10}\NormalTok{),}
          \AttributeTok{plot.title =} \FunctionTok{element\_text}\NormalTok{(}\AttributeTok{size=}\DecValTok{18}\NormalTok{, }\AttributeTok{hjust =} \FloatTok{0.5}\NormalTok{), }
          \AttributeTok{axis.title.y =} \FunctionTok{element\_text}\NormalTok{(}\AttributeTok{size =} \DecValTok{14}\NormalTok{))}
 



\NormalTok{device\_user\_percentage}
\end{Highlighting}
\end{Shaded}

\begin{verbatim}
## # A tibble: 4 x 4
##   Recording_Frequency steps_Percent sleep_Percent weight_percent
##   <chr>               <chr>         <chr>         <chr>         
## 1 High                76%           36%           6%            
## 2 Low                 4%            28%           18%           
## 3 Moderate            22%           10%           0%            
## 4 Never               0%            28%           76%
\end{verbatim}

\begin{Shaded}
\begin{Highlighting}[]
\NormalTok{actvties\_analysis\_graph}
\end{Highlighting}
\end{Shaded}

\begin{verbatim}
## Warning: `spread_()` was deprecated in tidyr 1.2.0.
## i Please use `spread()` instead.
## i The deprecated feature was likely used in the ggalluvial package.
##   Please report the issue at
##   <]8;;https://github.com/corybrunson/ggalluvial/issueshttps://github.com/corybrunson/ggalluvial/issues]8;;>.
\end{verbatim}

\begin{verbatim}
## Warning: The `.dots` argument of `group_by()` is deprecated as of dplyr 1.0.0.
## i The deprecated feature was likely used in the dplyr package.
##   Please report the issue at <]8;;https://github.com/tidyverse/dplyr/issueshttps://github.com/tidyverse/dplyr/issues]8;;>.
\end{verbatim}

\includegraphics{Bellabeat-CaseStudy_files/figure-latex/Device Engagement-1.pdf}

\begin{Shaded}
\begin{Highlighting}[]
\CommentTok{\#cleaning the environment, dropping all the obsolete dataframes}
\FunctionTok{rm}\NormalTok{(actvities\_graph\_data, device\_user\_percentage, actvties\_analysis\_graph)}
\end{Highlighting}
\end{Shaded}

\textbf{Key Insights:}

\begin{itemize}
\tightlist
\item
  \textbf{75\%+} users recorded their daily steps for more then 20days,
  this reduces to \textbf{36\%} when it comes to sleep monitoring and
  its further reduced to \textbf{6\%} when it comes to recording the
  weights.
\item
  Another trend which can be visualized here is of \textbf{Moderate
  Users},(those who recorded more then 10day and less then 20days), it
  keep decreasing i.e.~\textbf{22\%, 10\%, \& 0\%} for Steps group,
  Sleep Group and weight group respectively.
\item
  The data shares the trend that there is a sharp growth in \emph{Never
  Recorder's} chunk, as its \textbf{0\%} for Daily steps recorders, its
  \textbf{28\%} and \textbf{76\%} for Sleep and Weight recorders
  respectively.\\
  \strut \\
\end{itemize}

\hypertarget{daily-steps-records-vs-device-usage---scatter-plot}{%
\subparagraph{\texorpdfstring{\textbf{Daily Steps Records Vs Device
Usage - Scatter
plot}}{Daily Steps Records Vs Device Usage - Scatter plot}}\label{daily-steps-records-vs-device-usage---scatter-plot}}

We have to find out, is there a relationship between average activity
per day(daily steps) and recording device. To understand the
relationship clearly we will only be considering the users with more
then 10 days of steps recording.

\begin{Shaded}
\begin{Highlighting}[]
\CommentTok{\#filtering data for outliers}
\NormalTok{filter\_records }\OtherTok{\textless{}{-}} \FunctionTok{filter}\NormalTok{(daily\_steps\_records, total\_count }\SpecialCharTok{\textless{}} \DecValTok{10}\NormalTok{)}


\CommentTok{\#removed id "4057192912" because they only recorded 3 out of 31 days}

\NormalTok{  daily\_steps\_records }\OtherTok{\textless{}{-}} \FunctionTok{subset}\NormalTok{(daily\_steps\_records, id}\SpecialCharTok{!=} \DecValTok{4057192912}\NormalTok{)}

\CommentTok{\#verifying the data accuracy}
  \FunctionTok{filter}\NormalTok{(daily\_steps\_records, (daily\_steps\_records}\SpecialCharTok{$}\NormalTok{id }\SpecialCharTok{==} \StringTok{"4057192912"}\NormalTok{))}
\end{Highlighting}
\end{Shaded}

\begin{verbatim}
## # A tibble: 0 x 5
## # ... with 5 variables: id <dbl>, total_count <int>, steps_groups <chr>,
## #   sleep_groups <chr>, weight_groups <chr>
\end{verbatim}

\begin{Shaded}
\begin{Highlighting}[]
  \CommentTok{\#Recording Frequency vs Monthly Steps                                                         }
\NormalTok{device\_engagment }\OtherTok{\textless{}{-}} \FunctionTok{ggplot}\NormalTok{(}\AttributeTok{data=}\NormalTok{monthly\_user\_activity, }\FunctionTok{aes}\NormalTok{(}\AttributeTok{x =}\NormalTok{ record\_count, }\AttributeTok{y =}\NormalTok{ avg\_steps\_monthly))}\SpecialCharTok{+}
                    \FunctionTok{geom\_point}\NormalTok{() }\SpecialCharTok{+}
                    \FunctionTok{geom\_smooth}\NormalTok{(}\AttributeTok{method =}\NormalTok{ lm) }\SpecialCharTok{+}
                    \FunctionTok{theme\_minimal}\NormalTok{() }\SpecialCharTok{+}
                    \FunctionTok{labs}\NormalTok{(}\AttributeTok{title =} \StringTok{"Recording Frequency vs Monthly Steps"}\NormalTok{, }
                         \AttributeTok{x =} \StringTok{"Number of Days Recorded"}\NormalTok{, }\AttributeTok{y =} \StringTok{"Average Steps Per Participant"}\NormalTok{) }\SpecialCharTok{+}
                    \FunctionTok{theme}\NormalTok{(}\AttributeTok{plot.title =} \FunctionTok{element\_text}\NormalTok{(}\AttributeTok{size =} \DecValTok{12}\NormalTok{, }\AttributeTok{hjust =} \FloatTok{0.5}\NormalTok{),}
                          \AttributeTok{axis.title.x =} \FunctionTok{element\_text}\NormalTok{(}\AttributeTok{size =} \DecValTok{10}\NormalTok{),}
                          \AttributeTok{axis.title.y =} \FunctionTok{element\_text}\NormalTok{(}\AttributeTok{size =} \DecValTok{10}\NormalTok{)) }

\NormalTok{daily\_steps\_records}
\end{Highlighting}
\end{Shaded}

\begin{verbatim}
## # A tibble: 32 x 5
##            id total_count steps_groups sleep_groups weight_groups
##         <dbl>       <int> <chr>        <chr>        <chr>        
##  1 1503960366          30 High         High         Low          
##  2 1624580081          30 High         Never        Never        
##  3 1644430081          30 High         Low          Never        
##  4 1844505072          21 High         Low          Never        
##  5 1927972279          17 Moderate     Low          Low          
##  6 2022484408          31 High         Never        Never        
##  7 2026352035          31 High         High         Never        
##  8 2320127002          31 High         Low          Never        
##  9 2347167796          18 Moderate     Moderate     Never        
## 10 2873212765          31 High         Never        Low          
## # ... with 22 more rows
\end{verbatim}

\begin{Shaded}
\begin{Highlighting}[]
\NormalTok{device\_engagment}
\end{Highlighting}
\end{Shaded}

\begin{verbatim}
## `geom_smooth()` using formula 'y ~ x'
\end{verbatim}

\includegraphics{Bellabeat-CaseStudy_files/figure-latex/Daily Steps Records Vs Device Usage-1.pdf}

\begin{Shaded}
\begin{Highlighting}[]
\CommentTok{\#cleaning the environment, dropping all the obsolete dataframes}
\FunctionTok{rm}\NormalTok{(filter\_records,device\_engagment, daily\_steps\_records )}
\end{Highlighting}
\end{Shaded}

There is just one user with less than 10days records, renaming users
have recorded the steps for more than 10days. Its clear in above graph
that there is a positive correlation between daily steps and days
recorded, so we can say that the more someone records their health data,
we see a higher average daily steps.

\hfill\break
\hfill\break

\hypertarget{trends-in-time}{%
\subparagraph{\texorpdfstring{\textbf{Trends in
Time}}{Trends in Time}}\label{trends-in-time}}

Here, I'll be looking for answers of following questions. - Is there a
trend in time? - Is a particular time of the day or night when
participants use the device more? - who are the device users ? User
social profile if possible. - Most active hours ?

\begin{Shaded}
\begin{Highlighting}[]
\CommentTok{\#Most Active Hours}
                        
\NormalTok{steps\_perhour}\SpecialCharTok{$}\NormalTok{hours }\OtherTok{\textless{}{-}} \FunctionTok{format}\NormalTok{(}\FunctionTok{as.POSIXct}\NormalTok{(steps\_perhour}\SpecialCharTok{$}\NormalTok{hours), }\AttributeTok{format =} \StringTok{"\%H:\%M"}\NormalTok{)  }\CommentTok{\#remove the seconds and just have hours and minutes}

\NormalTok{ most\_active\_hours }\OtherTok{\textless{}{-}}\NormalTok{ steps\_perhour }\SpecialCharTok{\%\textgreater{}\%}
   \FunctionTok{group\_by}\NormalTok{(hours) }\SpecialCharTok{\%\textgreater{}\%}
   \FunctionTok{summarize}\NormalTok{(}\AttributeTok{avg\_steps\_hourly =} \FunctionTok{mean}\NormalTok{(step\_total)) }\SpecialCharTok{\%\textgreater{}\%} \CommentTok{\#find average steps for each hour}
   \FunctionTok{ggplot}\NormalTok{(}\FunctionTok{aes}\NormalTok{(hours, avg\_steps\_hourly, }\AttributeTok{group =} \DecValTok{1}\NormalTok{)) }\SpecialCharTok{+}
   \FunctionTok{geom\_line}\NormalTok{(}\AttributeTok{color =} \StringTok{"\#900C3F"}\NormalTok{) }\SpecialCharTok{+}
   \FunctionTok{ggtitle}\NormalTok{(}\StringTok{"Average Steps Per Hour"}\NormalTok{) }\SpecialCharTok{+}
   \FunctionTok{theme\_minimal}\NormalTok{() }\SpecialCharTok{+}
   \FunctionTok{theme}\NormalTok{(}\AttributeTok{axis.text.x =} \FunctionTok{element\_text}\NormalTok{(}\AttributeTok{angle =} \DecValTok{90}\NormalTok{, }\AttributeTok{vjust =} \FloatTok{0.5}\NormalTok{, }\AttributeTok{hjust=}\DecValTok{1}\NormalTok{),}
         \AttributeTok{text =} \FunctionTok{element\_text}\NormalTok{(}\AttributeTok{size =} \DecValTok{8}\NormalTok{)) }\SpecialCharTok{+}
   \FunctionTok{xlab}\NormalTok{(}\StringTok{"Hour of the Day"}\NormalTok{) }\SpecialCharTok{+} \FunctionTok{ylab}\NormalTok{(}\StringTok{"Average Step Count"}\NormalTok{) }\SpecialCharTok{+}
   \FunctionTok{annotate}\NormalTok{(}\StringTok{"rect"}\NormalTok{, }\AttributeTok{xmin =} \StringTok{"11:00"}\NormalTok{, }\AttributeTok{xmax =} \StringTok{"15:00"}\NormalTok{, }
            \AttributeTok{ymin =} \DecValTok{0}\NormalTok{, }\AttributeTok{ymax =} \DecValTok{700}\NormalTok{, }\AttributeTok{alpha =}\NormalTok{ .}\DecValTok{1}\NormalTok{) }\SpecialCharTok{+}
   \FunctionTok{annotate}\NormalTok{(}\StringTok{"rect"}\NormalTok{, }\AttributeTok{xmin =} \StringTok{"17:00"}\NormalTok{, }\AttributeTok{xmax =} \StringTok{"20:00"}\NormalTok{, }
            \AttributeTok{ymin =} \DecValTok{0}\NormalTok{, }\AttributeTok{ymax =} \DecValTok{700}\NormalTok{, }\AttributeTok{alpha =}\NormalTok{ .}\DecValTok{1}\NormalTok{) }\SpecialCharTok{+}
   \FunctionTok{annotate}\NormalTok{(}\StringTok{"rect"}\NormalTok{,  }\AttributeTok{xmin =} \StringTok{"07:00"}\NormalTok{, }\AttributeTok{xmax =} \StringTok{"10:00"}\NormalTok{, }
            \AttributeTok{ymin =} \DecValTok{0}\NormalTok{, }\AttributeTok{ymax =} \DecValTok{700}\NormalTok{, }\AttributeTok{alpha =}\NormalTok{ .}\DecValTok{1}\NormalTok{) }\SpecialCharTok{+}
      \FunctionTok{annotate}\NormalTok{(}\StringTok{"text"}\NormalTok{, }\AttributeTok{x =} \StringTok{"13:00"}\NormalTok{, }\AttributeTok{y =} \DecValTok{650}\NormalTok{, }
            \AttributeTok{label =} \StringTok{"Afternoon"}\NormalTok{, }\AttributeTok{hjust =} \StringTok{"center"}\NormalTok{, }\AttributeTok{size =} \DecValTok{3}\NormalTok{) }\SpecialCharTok{+}
   \FunctionTok{annotate}\NormalTok{(}\StringTok{"text"}\NormalTok{, }\AttributeTok{x =} \FloatTok{19.5}\NormalTok{, }\AttributeTok{y =} \DecValTok{650}\NormalTok{, }
            \AttributeTok{label =} \StringTok{"Evening"}\NormalTok{, }\AttributeTok{hjust =} \StringTok{"center"}\NormalTok{, }\AttributeTok{size =} \DecValTok{3}\NormalTok{) }\SpecialCharTok{+}
    \FunctionTok{annotate}\NormalTok{(}\StringTok{"text"}\NormalTok{, }\AttributeTok{x =} \FloatTok{9.5}\NormalTok{, }\AttributeTok{y =} \DecValTok{650}\NormalTok{,}
          \AttributeTok{label =} \StringTok{"Morning"}\NormalTok{, }\AttributeTok{hjust =} \StringTok{"center"}\NormalTok{, }\AttributeTok{size =} \DecValTok{3}\NormalTok{)}

 

\NormalTok{ steps\_perhour}
\end{Highlighting}
\end{Shaded}

\begin{verbatim}
## # A tibble: 22,099 x 4
##            id activity_hour hours step_total
##         <dbl> <date>        <chr>      <dbl>
##  1 1503960366 2016-04-12    00:00        373
##  2 1503960366 2016-04-12    01:00        160
##  3 1503960366 2016-04-12    02:00        151
##  4 1503960366 2016-04-12    03:00          0
##  5 1503960366 2016-04-12    04:00          0
##  6 1503960366 2016-04-12    05:00          0
##  7 1503960366 2016-04-12    06:00          0
##  8 1503960366 2016-04-12    07:00          0
##  9 1503960366 2016-04-12    08:00        250
## 10 1503960366 2016-04-12    09:00       1864
## # ... with 22,089 more rows
\end{verbatim}

\begin{Shaded}
\begin{Highlighting}[]
\NormalTok{ most\_active\_hours}
\end{Highlighting}
\end{Shaded}

\includegraphics{Bellabeat-CaseStudy_files/figure-latex/Trends in Time: Most Active Hours-1.pdf}

The above time trends show us that mostly there are a 3 type of active
users, the ones who exercise during morning time, some do it in
afternoon while there is peak in evening time as well. To analyze it
further we must break the data in quartiles and investigate.\\
\strut \\

\begin{Shaded}
\begin{Highlighting}[]
\NormalTok{ monthly\_user\_activity}\SpecialCharTok{$}\NormalTok{sum\_quartile }\OtherTok{\textless{}{-}} \FunctionTok{as.factor}\NormalTok{(}\FunctionTok{ntile}\NormalTok{(monthly\_user\_activity}\SpecialCharTok{$}\NormalTok{sum\_steps\_monthly, }\DecValTok{4}\NormalTok{))}
 
\NormalTok{ steps\_perhour }\OtherTok{\textless{}{-}} \FunctionTok{inner\_join}\NormalTok{(steps\_perhour, monthly\_user\_activity[ , }\FunctionTok{c}\NormalTok{(}\StringTok{"id"}\NormalTok{, }\StringTok{"sum\_quartile"}\NormalTok{)], }\AttributeTok{by =} \StringTok{"id"}\NormalTok{)}
 
 

\NormalTok{ most\_active\_hours }\OtherTok{\textless{}{-}}\NormalTok{  steps\_perhour }\SpecialCharTok{\%\textgreater{}\%}
   \FunctionTok{group\_by}\NormalTok{(sum\_quartile, hours) }\SpecialCharTok{\%\textgreater{}\%}
   \FunctionTok{summarize}\NormalTok{(}\AttributeTok{avg\_steps\_hourly =} \FunctionTok{mean}\NormalTok{(step\_total)) }\SpecialCharTok{\%\textgreater{}\%} \CommentTok{\#find average steps for each hour}
   \FunctionTok{ggplot}\NormalTok{(}\FunctionTok{aes}\NormalTok{(hours, avg\_steps\_hourly, }\AttributeTok{group =}\NormalTok{ sum\_quartile, }\AttributeTok{color =}\NormalTok{ sum\_quartile)) }\SpecialCharTok{+}
   \FunctionTok{geom\_line}\NormalTok{() }\SpecialCharTok{+}
   \FunctionTok{ggtitle}\NormalTok{(}\StringTok{"Average Steps Per Hour"}\NormalTok{) }\SpecialCharTok{+}
   \FunctionTok{theme\_minimal}\NormalTok{() }\SpecialCharTok{+}
   \FunctionTok{theme}\NormalTok{(}\AttributeTok{axis.text.x =} \FunctionTok{element\_text}\NormalTok{(}\AttributeTok{angle =} \DecValTok{90}\NormalTok{, }\AttributeTok{vjust =} \FloatTok{0.5}\NormalTok{, }\AttributeTok{hjust=}\DecValTok{1}\NormalTok{))}\SpecialCharTok{+}
   \FunctionTok{xlab}\NormalTok{(}\StringTok{"Hour of the Day"}\NormalTok{) }\SpecialCharTok{+} \FunctionTok{ylab}\NormalTok{(}\StringTok{"Average Steps"}\NormalTok{) }\SpecialCharTok{+}
   \FunctionTok{labs}\NormalTok{(}\AttributeTok{color =} \StringTok{"Quartile"}\NormalTok{) }\SpecialCharTok{+}
   \FunctionTok{annotate}\NormalTok{(}\StringTok{"rect"}\NormalTok{, }\AttributeTok{xmin =} \StringTok{"16:00"}\NormalTok{, }\AttributeTok{xmax =} \StringTok{"20:00"}\NormalTok{, }
            \AttributeTok{ymin =} \DecValTok{0}\NormalTok{, }\AttributeTok{ymax =} \DecValTok{1250}\NormalTok{, }\AttributeTok{alpha =}\NormalTok{ .}\DecValTok{1}\NormalTok{) }\SpecialCharTok{+}
   \FunctionTok{annotate}\NormalTok{(}\StringTok{"rect"}\NormalTok{, }\AttributeTok{xmin =} \StringTok{"11:00"}\NormalTok{, }\AttributeTok{xmax =} \StringTok{"15:00"}\NormalTok{, }
            \AttributeTok{ymin =} \DecValTok{0}\NormalTok{, }\AttributeTok{ymax =} \DecValTok{1250}\NormalTok{, }\AttributeTok{alpha =}\NormalTok{ .}\DecValTok{1}\NormalTok{) }\SpecialCharTok{+}
   \FunctionTok{annotate}\NormalTok{(}\StringTok{"rect"}\NormalTok{, }\AttributeTok{xmin =} \StringTok{"07:00"}\NormalTok{, }\AttributeTok{xmax =} \StringTok{"10:00"}\NormalTok{, }
            \AttributeTok{ymin =} \DecValTok{0}\NormalTok{, }\AttributeTok{ymax =} \DecValTok{1250}\NormalTok{, }\AttributeTok{alpha =}\NormalTok{ .}\DecValTok{1}\NormalTok{) }\SpecialCharTok{+}
   \FunctionTok{annotate}\NormalTok{(}\StringTok{"text"}\NormalTok{, }\AttributeTok{x =} \StringTok{"18:00"}\NormalTok{, }\AttributeTok{y =} \DecValTok{1210}\NormalTok{, }
            \AttributeTok{label =} \StringTok{"Evening"}\NormalTok{, }\AttributeTok{hjust =} \StringTok{"center"}\NormalTok{, }\AttributeTok{size =} \FloatTok{2.5}\NormalTok{) }\SpecialCharTok{+}
   \FunctionTok{annotate}\NormalTok{(}\StringTok{"text"}\NormalTok{, }\AttributeTok{x =} \StringTok{"13:00"}\NormalTok{, }\AttributeTok{y =} \DecValTok{1210}\NormalTok{, }
            \AttributeTok{label =} \StringTok{"Afternoon"}\NormalTok{, }\AttributeTok{hjust =} \StringTok{"center"}\NormalTok{, }\AttributeTok{size =} \FloatTok{2.5}\NormalTok{) }\SpecialCharTok{+}
   \FunctionTok{annotate}\NormalTok{(}\StringTok{"text"}\NormalTok{, }\AttributeTok{x =} \FloatTok{9.5}\NormalTok{, }\AttributeTok{y =} \DecValTok{1210}\NormalTok{, }
            \AttributeTok{label =} \StringTok{"Morning"}\NormalTok{, }\AttributeTok{hjust =} \StringTok{"center"}\NormalTok{, }\AttributeTok{size =} \FloatTok{2.5}\NormalTok{)}
\end{Highlighting}
\end{Shaded}

\begin{verbatim}
## `summarise()` has grouped output by 'sum_quartile'. You can override using the
## `.groups` argument.
\end{verbatim}

\begin{Shaded}
\begin{Highlighting}[]
\NormalTok{ most\_active\_hours}
\end{Highlighting}
\end{Shaded}

\includegraphics{Bellabeat-CaseStudy_files/figure-latex/Trends in Time: Most Active Hour - Quartiles-1.pdf}

\begin{Shaded}
\begin{Highlighting}[]
 \CommentTok{\#cleaning the environment, dropping the obsolete dataframe}
   \FunctionTok{rm}\NormalTok{(most\_active\_hours)}
\end{Highlighting}
\end{Shaded}

\textbf{Key Insights:} - We see that the top two most active groups( 3
and 4) of this sample peaks three times: around morning, afternoon and
evening, it seems these two groups represent the 9-5 office working
persons. Quartile 4 , have a small peak around 9:00am, and second peak
around 2:00pm while its highest/busiest peak hour is 7:00pm. Quartile 3,
its peak is around 8:00am, second peakis around 12:00pm, while its
busiest hour is around 5:00pm - Quartile 2, one of the least active
group has a small peak in the morning time, but it can be observed that
the afternoon and evening peaks as stronger and clearer. - Quartile 1,
the least active of the sample seems to have one small peak in the late
morning.\\
\strut \\
\strut \\

\hypertarget{sleep-trends}{%
\subparagraph{\texorpdfstring{\textbf{Sleep
Trends}}{Sleep Trends}}\label{sleep-trends}}

Lets look at the relationship of high intense activity time and sleep.
We need look for any trends that those who are more active throughout
the day sleep more than those who are less active? Could the duration of
high activity have an impact on this relationship?

I wouldn't be able to determine the precise difference between those two
variables (intense workout and bedtime) because I don't know what time
the person went to sleep. However, I can use the hour of maximum steps
as a color code to indicate how much time they slept that day in
relation to the number of steps they took.

\begin{Shaded}
\begin{Highlighting}[]
  \FunctionTok{colnames}\NormalTok{(sleep\_data)[}\DecValTok{2}\NormalTok{] }\OtherTok{\textless{}{-}} \StringTok{\textquotesingle{}date\textquotesingle{}} \CommentTok{\#renaming col2 from sleep\_data to date}

 \CommentTok{\#calculate hour of max steps taken}
\NormalTok{ max\_steps\_perhour }\OtherTok{\textless{}{-}}\NormalTok{ steps\_perhour }\SpecialCharTok{\%\textgreater{}\%}
   \FunctionTok{group\_by}\NormalTok{(id, activity\_hour) }\SpecialCharTok{\%\textgreater{}\%}
   \FunctionTok{summarise}\NormalTok{(}\AttributeTok{max\_step\_perhour =}\NormalTok{ hours[}\FunctionTok{which.max}\NormalTok{(step\_total)])}
\end{Highlighting}
\end{Shaded}

\begin{verbatim}
## `summarise()` has grouped output by 'id'. You can override using the `.groups`
## argument.
\end{verbatim}

\begin{Shaded}
\begin{Highlighting}[]
 \CommentTok{\#change to integer so it is accepted in the legend}
\NormalTok{ max\_steps\_perhour}\SpecialCharTok{$}\NormalTok{max\_step\_perhour }\OtherTok{\textless{}{-}} \FunctionTok{as.numeric}\NormalTok{(}\FunctionTok{gsub}\NormalTok{(}\StringTok{"(.*?)}\SpecialCharTok{\textbackslash{}\textbackslash{}}\StringTok{s|}\SpecialCharTok{\textbackslash{}\textbackslash{}}\StringTok{:00*"}\NormalTok{, }\StringTok{""}\NormalTok{, max\_steps\_perhour}\SpecialCharTok{$}\NormalTok{max\_step\_perhour))}

  \CommentTok{\#add total steps per day for each night of sleep}
\NormalTok{ sleep\_data }\OtherTok{\textless{}{-}} \FunctionTok{inner\_join}\NormalTok{(sleep\_data, daily\_activity[ , }\FunctionTok{c}\NormalTok{(}\StringTok{"id"}\NormalTok{, }\StringTok{"activity\_date"}\NormalTok{, }\StringTok{"total\_steps"}\NormalTok{)], }\AttributeTok{by =} \FunctionTok{c}\NormalTok{(}\StringTok{"id"} \OtherTok{=} \StringTok{"id"}\NormalTok{, }\StringTok{"date"} \OtherTok{=} \StringTok{"activity\_date"}\NormalTok{))}
 
\NormalTok{ sleep\_data }\OtherTok{\textless{}{-}} \FunctionTok{inner\_join}\NormalTok{(sleep\_data, max\_steps\_perhour, }\AttributeTok{by =} \FunctionTok{c}\NormalTok{(}\StringTok{"id"} \OtherTok{=} \StringTok{"id"}\NormalTok{, }\StringTok{"date"} \OtherTok{=} \StringTok{"activity\_hour"}\NormalTok{))}
 

\NormalTok{ sleep\_vs\_Steps\_graph }\OtherTok{\textless{}{-}}\NormalTok{ sleep\_data }\SpecialCharTok{\%\textgreater{}\%} 
   \FunctionTok{ggplot}\NormalTok{(}\FunctionTok{aes}\NormalTok{(}\AttributeTok{x =}\NormalTok{ total\_steps, }\AttributeTok{y =}\NormalTok{ total\_minutes\_asleep, }\AttributeTok{color =}\NormalTok{ max\_step\_perhour)) }\SpecialCharTok{+}
   \FunctionTok{geom\_point}\NormalTok{() }\SpecialCharTok{+}
   \FunctionTok{scale\_color\_gradientn}\NormalTok{(}\AttributeTok{colours =} \FunctionTok{rainbow}\NormalTok{(}\DecValTok{5}\NormalTok{),}
                         \AttributeTok{breaks =} \FunctionTok{c}\NormalTok{(}\FunctionTok{seq}\NormalTok{(}\DecValTok{0}\NormalTok{, }\DecValTok{20}\NormalTok{, }\DecValTok{5}\NormalTok{)),}
                         \AttributeTok{labels =} \FunctionTok{c}\NormalTok{(}\FunctionTok{sprintf}\NormalTok{(}
                           \StringTok{"\%s:00"}\NormalTok{,}
                           \FunctionTok{seq}\NormalTok{(}\DecValTok{0}\NormalTok{, }\DecValTok{20}\NormalTok{, }\DecValTok{5}\NormalTok{)}
\NormalTok{                         ))) }\SpecialCharTok{+}
   \FunctionTok{geom\_smooth}\NormalTok{(}\AttributeTok{method =}\NormalTok{ lm, }\AttributeTok{alpha =} \FloatTok{0.3}\NormalTok{) }\SpecialCharTok{+}
   \FunctionTok{theme\_linedraw}\NormalTok{() }\SpecialCharTok{+}
   \FunctionTok{labs}\NormalTok{(}\AttributeTok{title =} \StringTok{"Sleep vs Steps (per day)"}\NormalTok{, }
        \AttributeTok{x =} \StringTok{"Minutes Asleep"}\NormalTok{, }\AttributeTok{y =} \StringTok{"Steps"}\NormalTok{,}
        \AttributeTok{color =} \StringTok{"Hour of Max Step"}\NormalTok{) }\SpecialCharTok{+}
   \FunctionTok{theme}\NormalTok{(}\AttributeTok{plot.title =} \FunctionTok{element\_text}\NormalTok{(}\AttributeTok{size =} \DecValTok{12}\NormalTok{, }\AttributeTok{hjust =} \FloatTok{0.5}\NormalTok{),}
         \AttributeTok{axis.title.x =} \FunctionTok{element\_text}\NormalTok{(}\AttributeTok{size =} \DecValTok{10}\NormalTok{),}
         \AttributeTok{axis.title.y =} \FunctionTok{element\_text}\NormalTok{(}\AttributeTok{size =} \DecValTok{10}\NormalTok{),}
         \AttributeTok{legend.key.size =} \FunctionTok{unit}\NormalTok{(}\DecValTok{5}\NormalTok{, }\StringTok{\textquotesingle{}mm\textquotesingle{}}\NormalTok{),}
         \AttributeTok{legend.title =} \FunctionTok{element\_text}\NormalTok{(}\AttributeTok{size =} \DecValTok{8}\NormalTok{),}
         \AttributeTok{legend.text =} \FunctionTok{element\_text}\NormalTok{(}\AttributeTok{size =} \DecValTok{6}\NormalTok{)) }
 
\NormalTok{ sleep\_vs\_Steps\_graph}
\end{Highlighting}
\end{Shaded}

\begin{verbatim}
## `geom_smooth()` using formula 'y ~ x'
\end{verbatim}

\includegraphics{Bellabeat-CaseStudy_files/figure-latex/Steps Vs Sleep-1.pdf}

\begin{Shaded}
\begin{Highlighting}[]
  \CommentTok{\#cleaning the environment, dropping the obsolete dataframe}
   \FunctionTok{rm}\NormalTok{(sleep\_vs\_Steps\_graph, max\_steps\_perhour)}
\end{Highlighting}
\end{Shaded}

Here, Sleep and steps is compared for each participant, above scatter
plot shows the comparison between total minutes asleep and total steps
taken. In order to estimate whether a participant's most intense workout
had an impact on their ability to sleep, I colored the hour in which
they took the most steps. Surprisingly, a negative relationship between
sleep and daily steps is revealed.This sample shows that as a person
walks more, his/her sleep gets reduced.\\

Now, we have to look for any trends in days, do any particular
day(i.e.~weekends) effects the activities or users have the same routine
throughout the week.

\begin{Shaded}
\begin{Highlighting}[]
\NormalTok{ sleep\_data}\SpecialCharTok{$}\NormalTok{difference\_bedtime }\OtherTok{\textless{}{-}}\NormalTok{ sleep\_data}\SpecialCharTok{$}\NormalTok{total\_time\_in\_bed }\SpecialCharTok{{-}}\NormalTok{sleep\_data}\SpecialCharTok{$}\NormalTok{total\_minutes\_asleep}
 
 \FunctionTok{mean}\NormalTok{(sleep\_data}\SpecialCharTok{$}\NormalTok{total\_time\_in\_bed }\SpecialCharTok{{-}}\NormalTok{ sleep\_data}\SpecialCharTok{$}\NormalTok{total\_minutes\_asleep)}
\end{Highlighting}
\end{Shaded}

\begin{verbatim}
## [1] 39.44963
\end{verbatim}

\begin{Shaded}
\begin{Highlighting}[]
 \CommentTok{\#prepping the data for histogram (long format)}
\NormalTok{ bedtime\_histogram }\OtherTok{\textless{}{-}} 
   \FunctionTok{inner\_join}\NormalTok{(sleep\_data[}\FunctionTok{c}\NormalTok{(}\StringTok{"id"}\NormalTok{, }\StringTok{"total\_minutes\_asleep"}\NormalTok{, }\StringTok{"total\_time\_in\_bed"}\NormalTok{)], monthly\_user\_activity[}\FunctionTok{c}\NormalTok{(}\StringTok{"id"}\NormalTok{, }\StringTok{"sum\_quartile"}\NormalTok{)], }\AttributeTok{by =} \StringTok{"id"}\NormalTok{) }\SpecialCharTok{\%\textgreater{}\%}
   \FunctionTok{pivot\_longer}\NormalTok{(}\SpecialCharTok{!}\FunctionTok{c}\NormalTok{(id, sum\_quartile))}
 
 \CommentTok{\#summarizing data for labels}
\NormalTok{ mean\_difference }\OtherTok{\textless{}{-}}\NormalTok{ sleep\_data }\SpecialCharTok{\%\textgreater{}\%}
   \FunctionTok{summarise}\NormalTok{(}\AttributeTok{average\_difference =} \FunctionTok{round}\NormalTok{(}\FunctionTok{mean}\NormalTok{(difference\_bedtime), }\DecValTok{2}\NormalTok{),}
             \AttributeTok{data\_count =} \FunctionTok{n}\NormalTok{(),}
             \AttributeTok{group\_size =} \FunctionTok{length}\NormalTok{(}\FunctionTok{unique}\NormalTok{(id)))}
 

\NormalTok{    bedtime\_histogram}
\end{Highlighting}
\end{Shaded}

\begin{verbatim}
## # A tibble: 814 x 4
##            id sum_quartile name                 value
##         <dbl> <fct>        <chr>                <dbl>
##  1 1503960366 4            total_minutes_asleep   327
##  2 1503960366 4            total_time_in_bed      346
##  3 1503960366 4            total_minutes_asleep   384
##  4 1503960366 4            total_time_in_bed      407
##  5 1503960366 4            total_minutes_asleep   412
##  6 1503960366 4            total_time_in_bed      442
##  7 1503960366 4            total_minutes_asleep   340
##  8 1503960366 4            total_time_in_bed      367
##  9 1503960366 4            total_minutes_asleep   700
## 10 1503960366 4            total_time_in_bed      712
## # ... with 804 more rows
\end{verbatim}

\begin{Shaded}
\begin{Highlighting}[]
\NormalTok{    mean\_difference}
\end{Highlighting}
\end{Shaded}

\begin{verbatim}
## # A tibble: 1 x 3
##   average_difference data_count group_size
##                <dbl>      <int>      <int>
## 1               39.4        407         24
\end{verbatim}

\begin{Shaded}
\begin{Highlighting}[]
\NormalTok{     daygraph }\OtherTok{\textless{}{-}}\NormalTok{ daily\_activity}
\NormalTok{ daygraph }\OtherTok{\textless{}{-}}\NormalTok{ daygraph [ }\SpecialCharTok{{-}}\FunctionTok{c}\NormalTok{(}\DecValTok{1}\NormalTok{,}\DecValTok{2}\NormalTok{,}\DecValTok{5}\NormalTok{,}\DecValTok{6}\NormalTok{,}\DecValTok{7}\NormalTok{,}\DecValTok{8}\NormalTok{,}\DecValTok{9}\NormalTok{,}\DecValTok{10}\NormalTok{,}\DecValTok{11}\NormalTok{,}\DecValTok{12}\NormalTok{,}\DecValTok{13}\NormalTok{,}\DecValTok{14}\NormalTok{,}\DecValTok{15}\NormalTok{,}\DecValTok{16}\NormalTok{) ]}
 
\NormalTok{   daygraph }\OtherTok{\textless{}{-}}\NormalTok{ daygraph }\SpecialCharTok{\%\textgreater{}\%}
     \FunctionTok{group\_by}\NormalTok{(day) }\SpecialCharTok{\%\textgreater{}\%}
     \FunctionTok{summarise}\NormalTok{(}
     \AttributeTok{avg =} \FunctionTok{mean}\NormalTok{(total\_steps))}
   
   
 \CommentTok{\#cleaning the environment, dropping the obsolete dataframe}
 \FunctionTok{rm}\NormalTok{ ( mean\_difference)}
\end{Highlighting}
\end{Shaded}

\begin{Shaded}
\begin{Highlighting}[]
\FunctionTok{ggplot}\NormalTok{(daygraph, }\FunctionTok{aes}\NormalTok{(}\AttributeTok{x=} \FunctionTok{factor}\NormalTok{(day, }\AttributeTok{levels =} \FunctionTok{c}\NormalTok{(}\StringTok{"Monday"}\NormalTok{, }\StringTok{"Tuesday"}\NormalTok{, }\StringTok{"Wednesday"}\NormalTok{, }\StringTok{"Thursday"}\NormalTok{, }\StringTok{"Friday"}\NormalTok{, }\StringTok{"Saturday"}\NormalTok{, }\StringTok{"Sunday"}\NormalTok{)), }\AttributeTok{y=}\NormalTok{ avg)) }\SpecialCharTok{+}
  \FunctionTok{geom\_bar}\NormalTok{(}\AttributeTok{position=}\StringTok{"dodge"}\NormalTok{, }\AttributeTok{stat=}\StringTok{"identity"}\NormalTok{)}\SpecialCharTok{+}
  \FunctionTok{theme\_light}\NormalTok{()}\SpecialCharTok{+}
    \FunctionTok{labs}\NormalTok{(}\AttributeTok{title =} \StringTok{"Day to Day comparision"}\NormalTok{, }
       \AttributeTok{x =} \StringTok{"Weekdays"}\NormalTok{, }\AttributeTok{y =} \StringTok{"Steps"}\NormalTok{,}
       \AttributeTok{color =} \StringTok{"Average Steps"}\NormalTok{)}\SpecialCharTok{+}
  \FunctionTok{geom\_text}\NormalTok{(}\FunctionTok{aes}\NormalTok{(}\AttributeTok{label =} \FunctionTok{signif}\NormalTok{(avg, }\AttributeTok{digits =} \DecValTok{3}\NormalTok{)),  }\AttributeTok{vjust =} \FloatTok{1.5}\NormalTok{, }\AttributeTok{colour =} \StringTok{"white"}\NormalTok{)}
\end{Highlighting}
\end{Shaded}

\includegraphics{Bellabeat-CaseStudy_files/figure-latex/Day to Day comparision-1.pdf}

The above bar graphs show that there is no pattern that suggests a rise
or fall in user engagement, and user activities are unaffected by
weekends. There is nothing more that can be analyzed from the sample of
data provided; in order to conduct a more thorough analysis, more data
would be required. According to current data Tuesdays and Saturdays are
the busiest days.

\textbf{Sleep delays: Time in bed vs Time asleep} Now we need to analyze
how much time is taken by a users from going to bed to going to sleep.
we can analyze it by using the sleep dataframe. The average sleep delay
is about 40 minutes. lets visualize and analyze it..

\begin{Shaded}
\begin{Highlighting}[]
\CommentTok{\#Density Histogram of Time Asleep and In Bed}

\NormalTok{ bedtime\_histogram }\SpecialCharTok{\%\textgreater{}\%}
   \FunctionTok{ggplot}\NormalTok{(}\FunctionTok{aes}\NormalTok{(}\AttributeTok{x =}\NormalTok{ value, }\AttributeTok{color =}\NormalTok{ name, }\AttributeTok{fill =}\NormalTok{ name)) }\SpecialCharTok{+}
   \FunctionTok{geom\_histogram}\NormalTok{(}\FunctionTok{aes}\NormalTok{(}\AttributeTok{y =}\NormalTok{ ..density..),  }\AttributeTok{position =} \StringTok{"identity"}\NormalTok{, }\AttributeTok{alpha =}\NormalTok{ .}\DecValTok{5}\NormalTok{, }\AttributeTok{binwidth =} \DecValTok{35}\NormalTok{) }\SpecialCharTok{+} 
   \FunctionTok{guides}\NormalTok{(}\AttributeTok{col =} \ConstantTok{FALSE}\NormalTok{) }\SpecialCharTok{+}
   \FunctionTok{geom\_density}\NormalTok{(}\AttributeTok{alpha =} \DecValTok{0}\NormalTok{, }\AttributeTok{lwd =}\NormalTok{ .}\DecValTok{6}\NormalTok{) }\SpecialCharTok{+}
   \FunctionTok{theme\_light}\NormalTok{() }\SpecialCharTok{+}
     \FunctionTok{theme}\NormalTok{(}\AttributeTok{legend.position =} \StringTok{"top"}\NormalTok{,}
         \AttributeTok{legend.key.size =} \FunctionTok{unit}\NormalTok{(}\DecValTok{5}\NormalTok{, }\StringTok{\textquotesingle{}mm\textquotesingle{}}\NormalTok{),}
         \AttributeTok{text =} \FunctionTok{element\_text}\NormalTok{(}\AttributeTok{size =} \DecValTok{7}\NormalTok{))}\SpecialCharTok{+}
   \FunctionTok{labs}\NormalTok{(}\AttributeTok{title =} \StringTok{"Time in bed vs Time asleep "}\NormalTok{, }
        \AttributeTok{x =} \StringTok{"Minutes"}\NormalTok{, }\AttributeTok{y =} \StringTok{"Density"}\NormalTok{) }\SpecialCharTok{+}
   \FunctionTok{scale\_fill\_discrete}\NormalTok{(}\AttributeTok{name =} \StringTok{"Time (minutes)"}\NormalTok{, }\AttributeTok{labels =} \FunctionTok{c}\NormalTok{(}\StringTok{"Asleep"}\NormalTok{, }\StringTok{"In Bed"}\NormalTok{))}
\end{Highlighting}
\end{Shaded}

\begin{verbatim}
## Warning: `guides(<scale> = FALSE)` is deprecated. Please use `guides(<scale> =
## "none")` instead.
\end{verbatim}

\includegraphics{Bellabeat-CaseStudy_files/figure-latex/Sleep delays: Time in bed vs Time asleep-1.pdf}

\hfill\break
\hfill\break

\hypertarget{step-6-act}{%
\subsection{Step 6: Act}\label{step-6-act}}

\hypertarget{key-insight}{%
\paragraph{\texorpdfstring{\textbf{Key
Insight:}}{Key Insight:}}\label{key-insight}}

\textbf{Use of feature} - The most used feature is step count. - Second
most used feature is sleep monitoring. - The lease utilized feature is
weight recording. - If we compare all the features Steps\_count is used
by around 100\% population on regular basis, and weight\_record is the
least used feature by the user

\textbf{Time Trends} - Most Active days are Tuesdays and Saturdays and
least active days are Friday \& Sundays. - Most of the users use the
device for steps recording between following hours; 07:00-8:00am,
12:00pm-2:00pm and 5:00pm-7:00pm.

\textbf{Sleep Trends} - People in this sample slept for shorter periods
of time as they logged more steps. - The average sleep delay is about 40
minutes

\hypertarget{recommendations}{%
\paragraph{\texorpdfstring{\textbf{Recommendations:}}{Recommendations:}}\label{recommendations}}

\textbf{Competitions} - Create a regional leader board (like Ludo star)
with users' weekly/monthly step counts compared to one another for
competitions. The top 3 users can receive gifts or coupons for discounts
on future wearable purchases at the end of each term.

\textbf{Notifications} - Provide notifications to the user regrading
daily/weekly activity goal. In order to aid users in getting ready for
bed, notifications could also be sent an hour beforehand.

\textbf{Referrals} - Include a referral program so that existing
wearable users can get progressively bigger discounts on goods or future
wearable purchases based on how many of their referrals buy a wearable.

\textbf{User Engagement Feature} - develop some feature which will
encourage use of the product to further your fitness goals i.e.Point
system or Goals/streaks.

To conclude, I wanted to help Bellabeat better understand its customer
base and provide helpful advice so that customers can reach their
fitness objectives. Following my suggestions will significantly increase
customer engagement with our products and Bellabeat's long-term success.

\end{document}
